\documentclass[aspectratio=169]{beamer}
\geometry{paperwidth=160mm,paperheight=100mm}
\usepackage{beamerthemesidebar}
\usepackage{hyperref}
\usepackage{color}
\usepackage{multimedia}
\usepackage{colortbl}
\usepackage{amsmath}
\usepackage{empheq}
\usepackage{cancel}
\usepackage{amssymb}
\usepackage{amsfonts}
\usepackage{lipsum}
\usepackage{tcolorbox}
\usepackage{tabularx}
\usepackage{caption}
\usepackage{bm}

\setbeamersize{sidebar width right=0pt}
\setbeamertemplate{footline}[frame number]
%
\definecolor{orange}{RGB}{250,167,12}
\definecolor{yellow}{RGB}{246,250,12}
\definecolor{green}{RGB}{128,238,1}
\definecolor{black}{RGB}{0,0,0}
\definecolor{blue}{RGB}{0,0,255}
\definecolor{red}{RGB}{255,0,0}
\definecolor{sepia}{RGB}{94,38,18}
\newcommand{\ve}[1]{{\rm\bf {#1}}}
\newcommand{\q}[1]{\textcolor{blue}{#1}}
\newcommand{\blue}[1]{\textcolor{blue}{#1}}
\newcommand{\sepia}[1]{\textcolor{sepia}{#1}}
\newcommand{\red}[1]{\textcolor{red}{#1}}
\newcommand{\green}[1]{\textcolor{green}{#1}}
\newcommand{\yellow}[1]{\textcolor{yellow}{#1}}
\newcommand{\orange}[1]{\textcolor{orange}{#1}}
\definecolor{burlywood}{RGB}{255,211,155}
\definecolor{chocolate}{RGB}{255,127,36}
\definecolor{tan}{RGB}{210,180,140}
%
\def\onethird{{\textstyle{1\over3}}}
\def\twothirds{{\textstyle{2\over3}}}
\def\fourthirds{{\textstyle{4\over3}}}
\def\onehalf{{\textstyle{1\over2}}}
\def\threehalfs{{\textstyle{3\over2}}}
%
\newcommand{\pd}{\partial}
\newcommand{\aMLT}{\alpha_{\rm MLT}}
\newcommand{\Fconv}{F_{\rm conv}}
\newcommand{\Frad}{F_{\rm rad}}
\newcommand{\Ftot}{F_{\rm tot}}
\newcommand{\Hp}{H_p}
\newcommand{\prad}{p_{\rm rad}}
\newcommand{\pgas}{p_{\rm gas}}
\newcommand{\TTc}{T_{\rm c}}
\newcommand{\rhoc}{\rho_{\rm c}}
\newcommand{\Teff}{T_{\rm eff}}
\newcommand{\Fstar}{F_\star}
\newcommand{\pstar}{p_\star}
\newcommand{\Pstar}{P_\star}
\newcommand{\Rstar}{R_\star}
\newcommand{\rhostar}{\rho_\star}
\newcommand{\Tstar}{T_\star}
%
\title{Theoretical Astrophysics I: Physics of Sun and Stars\\
Lecture 9: Stellar Atmospheres}
\author{\texorpdfstring{\sepia{Petri K\"{a}pyl\"{a} Ivan Mili\'{c}}\newline\blue{\url{pkapyla, milic@leibniz-kis.de}}}{}}
\institute{Institut f\"ur Sonnenphysik - KIS, Freiburg}
\date{\today}
%
\begin{document}
\frame{\titlepage}


\section{Recap and intro}
%
\frame{
\frametitle{Recap}
\begin{itemize}
\item So far we have dealt with interiors of the stars - these contain an absolute majority of stellar material. 
\item We studied the equations that govern stellar structure and the evolution and solved them in more or less detail to understand how the stars look into the inside and how they evolve on the HR diagram. 
\item None of these processes were \textbf{directly} observable. The only observable quantity we talked about was the stellar luminosity ($L$). 
\item Now we are going to talk about the structures that we can observe - stellar atmospheres. 
\end{itemize}
}
%
%
\frame{
\frametitle{Why study stellar atmospheres?}
\begin{minipage}{0.59\linewidth}
\begin{itemize}
\item Stellar atmosphere is the miniscule part on the very right of this diagram. 
\item How can it be important? (Salpeter once asked the same question)
\item First: Stellar atmosphere is the only part we can observe. It gives us indirect insight into what is going on inside. 
\item Second: Stellar Atmospheres can influence the evolution of the star, at certain evolution stages atmospheres can be huge and mass loss happens there. 
\item Third: There is a lot of interesting physics going on (especially the case for the Sun, talk next week!)
\end{itemize}
\end{minipage}
\begin{minipage}{0.4\linewidth}
\begin{figure}
\includegraphics[width=5cm]{figures/solar_density.png}
\caption*{Credits: CD}
\end{figure}
\end{minipage}
}
%
%
\frame{
\frametitle{Differences between stellar atmospheres and stellar interiors}
\begin{itemize}
\item 
\end{itemize}
}
%
%
\frame{
\frametitle{Hayashi zone and pre-main-sequence phase}
\begin{itemize}
\item The photosphere needs to be able to radiate away all of the
  incoming energy flux. This is determined by the thermodynamic
  structure, i.e., the drop of $p$, $\rho$, and $T$ accross it.
\item In Hydrostatic equilibrium
  \begin{equation}
    \frac{dp}{dr} \approx -\rho \frac{GM}{R^2},
  \end{equation}
  which can be integrated from $R$ to the point where $p$ vanishes
  \begin{equation}
    p_R = \frac{GM}{R^2}\int_R^\infty \rho dr.\label{equ:pR1}
  \end{equation}
\item Furthermore, the optical depth of the photosphere, characterised
  by $\Teff$, is of the order on unity and thus $\int_R^\infty \kappa
  \rho dr = \overline{\kappa} \int_R^\infty \rho dr$, where
  $\overline{\kappa}$ is the mean opacity in the photosphere.
\item Taking $\overline{\kappa} = \kappa(R)$ and assuming it to be a
  power law in $\rho_R$ and $\Teff$ gives:
  \begin{equation}
    \kappa_0 \rho_R^a \Teff^b \int_R^\infty \rho dr = 1.\label{equ:optau}
  \end{equation}
\end{itemize}
}
%
%
\frame{
\frametitle{Hayashi zone and pre-main-sequence phase}
\begin{itemize}
\item Combining Eqs.~(\ref{equ:pR1}) and (\ref{equ:optau}) gives:
  \begin{equation}
    p_R = \frac{GM}{R^2 \kappa_0} \rho_R^{-a} \Teff^{-b}.\label{equ:pR2}
  \end{equation}
\item Yet another relation between the thermodynamic quantities at $R$
  is given by the equation of state, here taken to be ideal gas equation:
  \begin{equation}
    p_R = \frac{\cal R}{\mu} \rho_R \Teff.\label{equ:peos}
  \end{equation}
\item Finally, the temperature at $R$ is related to the luminosity via
  \begin{equation}
    L = 4\pi R^2 \Teff^4.\label{equ:Lumi}
  \end{equation}
\item Now we have four equations that describe the surface of the
  star: Eqs.(\ref{equ:ppoly}) (with Eqs.~(\ref{equ:Kn}) and
  (\ref{equ:Cn})), (\ref{equ:pR2}), (\ref{equ:peos}), and
  (\ref{equ:Lumi})
\end{itemize}
}
%
%
\frame{
\frametitle{Hayashi zone and pre-main-sequence phase}
\begin{itemize}
\item These read in logarithmic form:
  \begin{eqnarray}
    & n \log p_R = (n-1) \log M + (3-n)\log R + (n+1) \log \rho_r + \mbox{const.} & \\
    & \log p_R = \log M - 2 \log R - a \log \rho_r - b \log \Teff + \mbox{const.} & \\
    & \log p_R = \log \rho_R + \log \Teff + \mbox{const.} & \\
    & \log L = 2 \log R + 4 \log \Teff + \mbox{const.} &
  \end{eqnarray}
\item Eliminating $\log R$, $\log \rho_R$, and $\log p_R$ yields:
  \begin{eqnarray}
    & \log L = A \log \Teff + B \log M + \mbox{const.} & \\
    & A =  \frac{(7-n)(a+1)-4-a+b}{0.5(3-n)(a+1)-1}, \ \ B = -\frac{(n-1)(a+1)+1}{0.5(3-n)(a+1)-1}. &
  \end{eqnarray}
\item This relation traces the \emph{Hayashi track} in the HR
  diagram. These should not be interpreted as evolutionary tracks but
  rather as an asymptote.
\end{itemize}
}
%
%
\frame{
\frametitle{Hayashi zone and pre-main-sequence phase}
\begin{itemize}
\item We assume for simplicity that $a = 1$ which is reasonably
  accurate, such that
  \begin{eqnarray}
    & A =  \frac{9-2n+b}{2-n}, \ \ B = -\frac{2n-1}{2-n}. &
  \end{eqnarray}
  $b$ varies much more but is usually positive
\item Dynamical stability requires that $n<3$ and therefore the
  polytropic index is limited to the range $1.5 \leq n < 3$.
\item For $b\approx 4$ and $n=1.5$ we find that $A=20$. This means
  that the Hayashi track is almost vertical in the ($\log \Teff, \log
  L$) plane.
\item As a function of mass the tracks are stacked near each other and
  higher mass leads to a shift toward higher temperatures because $A$
  and $B$ have opposite signs.
\item The slope changes with composition that can be associated with
  an effective polytropic index.
\end{itemize}
}
%
%
\frame{
\frametitle{Hayashi zone and pre-main-sequence phase}
\begin{itemize}
\item The signifigance of the Hayashi track can be seen from
  considering $\overline{\gamma}$ which is an average value of $\gamma
  = \frac{d\ln p}{d\ln\rho}$ over the whole
  star. $\overline{\gamma}_{\rm a}$ is the corresponding adiabatic
  index.
\item For a fully convective star $\overline{\gamma} =
  \overline{\gamma}_{\rm a}$.
\item If any part of the star is radiative with $\gamma<\gamma_{\rm
  a}$, then $\overline{\gamma} < \overline{\gamma}_{\rm
  a}$. Correspondingly, the average polytropic index $n>n_{\rm a}$
  where $n_{\rm a}$ is the adiabatic polytropic index defining the
  Hayashi track.
\item If $\overline{\gamma} > \overline{\gamma}_{\rm a}$, the
  situation is unstable and therefore such state is ``forbidden''. In
  practise in such a situation, convection in the star would very
  quickly restore near-adiabaticity by transporting any excess heat to
  the surface because a very small superadiabaticity is enough to
  transport massive amounts of energy (homework!).
\end{itemize}
}
%
%
\frame{
\frametitle{Hayashi zone and pre-main-sequence phase}
\begin{minipage}{0.59\linewidth}
\begin{itemize}
\item Stars from from contrating gas clouds (molecular clouds) through
  dynamical collapse. These clouds are (parsecs) and fragment in the
  process.
\item Most of the gas in such clouds is in the form of molecular
  hydrogen (H$_2$). The collapse happens in dynamical timescale
  $\tau_{\rm dyn} \propto \rho^{-1.2}$.
\item Gradually the H$_2$ molecules are dissociated, after which
  hydrogen and later helium start to be ionised. These processes use
  up most of the energy from continuing collapse and the temperature
  stays nearly constant.
\item Finally the ionisation is nearly complete and the temperature
  starts to increase and a hydrostatic equilibrium is restored. The
  object is now a protostar.
\end{itemize}
\end{minipage}
\begin{minipage}{0.4\linewidth}
\begin{figure}
\includegraphics[width=5cm]{figures/Iben_1965_PMS.png}
\caption*{Credits: Iben (1965), Astrophys. J., 141}
\end{figure}
\end{minipage}
}
%
%
\frame{
\frametitle{Hayashi zone and pre-main-sequence phase}
\begin{minipage}{0.59\linewidth}
\begin{itemize}
\item Estimate of protostellar radius can be obtained by assuming that
  all of the gravitational energy is spent to dissociate H$_2$ and
  ionize H and He. Then,
  \begin{equation}
    \alpha \frac{GM^2}{R_{\rm ps}} \approx \frac{M}{m_{\rm H}}\left( \frac{X}{2}\chi_{{\rm H}_2} + X_{\chi_{\rm H}} + \frac{Y}{4}\chi_{\rm He} \right),
  \end{equation}
  where $\chi_{{\rm H}_2} =4.5$~eV, $\chi_{\rm H} = 13.6$~eV, and
  $\chi_{\rm He} = 79$~eV.
\item Taking $Y \approx 1 -X$ and $\alpha = \onehalf$ gives
  \begin{equation}
    \frac{R_{\rm ps}}{R_\odot} \approx \frac{50}{1-0.2X} \frac{M}{M_\odot}.
  \end{equation}
\end{itemize}
\end{minipage}
\begin{minipage}{0.4\linewidth}
\begin{figure}
\includegraphics[width=5cm]{figures/Iben_1965_PMS.png}
\caption*{Credits: Iben (1965), Astrophys. J., 141}
\end{figure}
\end{minipage}
}
%
%
\frame{
\frametitle{Hayashi zone and pre-main-sequence phase}
\begin{minipage}{0.59\linewidth}
\begin{itemize}
\item Recalling the average temperature from virial theorem and
  inserting the estimate for $R_{\rm ps}$ with $X=0.7$ gives:
  \begin{equation}
    \overline{T} = \frac{\alpha}{3} \frac{\mu}{k} \frac{GMm_{\rm H}}{R_{\rm ps}} \approx 6\cdot 10^4~{\rm K}.
  \end{equation}
  Note that the temperature is independent of $M$.
\item At this starting point on the Hayashi track the star is fully
  convective and the gas is still opaque.
\item Contraction continues until all of the gas is ionized. The
  opacity drops first in the interior and the convection zone recedes.
  $\Teff$ starts to rise slowly.
\item Nuclear reactions start gradually when core temperature
  increases and increase the luminosity. Evolutionary track is
  complicated by ignition of different branches of hydrogen burning.
\end{itemize}
\end{minipage}
\begin{minipage}{0.4\linewidth}
\begin{figure}
\includegraphics[width=5cm]{figures/Iben_1965_PMS.png}
\caption*{Credits: Iben (1965), Astrophys. J., 141}
\end{figure}
\end{minipage}
}
%
%
\frame{
\frametitle{Hayashi zone and pre-main-sequence phase}
\begin{minipage}{0.59\linewidth}
\begin{figure}
\includegraphics[width=5cm]{figures/Prialnik_PMS_lifetimes.png}
\caption*{Credits: Prialnik.}
\end{figure}
\begin{itemize}
\item The time that stars spend in the PMS phase depends strongly on
  the mass.
\end{itemize}
\end{minipage}
\begin{minipage}{0.4\linewidth}
\begin{figure}
\includegraphics[width=5cm]{figures/Iben_1965_PMS.png}
\caption*{Credits: Iben (1965), Astrophys. J., 141}
\end{figure}
\end{minipage}
}
%
\frame{
\frametitle{Main-sequence phase}
\begin{itemize}
\item TBA
\end{itemize}
}
%
%
\frame{
\frametitle{Main-sequence phase}
\begin{itemize}
\item TBA
\end{itemize}
}
%
%
\frame{
\frametitle{Main-sequence phase}
\begin{itemize}
\item TBA
\end{itemize}
}
%
%
\frame{
\frametitle{Red Giant phase}
\begin{itemize}
\item TBA
\end{itemize}
}
%
%
\frame{
\frametitle{Red Giant phase}
\begin{itemize}
\item TBA
\end{itemize}
}
%
%
\frame{
\frametitle{Red Giant phase}
\begin{itemize}
\item TBA
\end{itemize}
}
%
%
%
\end{document}
% 

