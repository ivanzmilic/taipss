\documentclass[aspectratio=169]{beamer}
\geometry{paperwidth=160mm,paperheight=100mm}
\usepackage{beamerthemesidebar}
\usepackage{hyperref}
\usepackage{color}
\usepackage{multimedia}
\usepackage{colortbl}
\usepackage{amsmath}
\usepackage{empheq}
\usepackage{cancel}
\usepackage{amssymb}
\usepackage{amsfonts}
\usepackage{lipsum}
\usepackage{tcolorbox}
\usepackage{tabularx}
\usepackage{caption}
\usepackage{bm}

\setbeamersize{sidebar width right=0pt}
\setbeamertemplate{footline}[frame number]
%
\definecolor{orange}{RGB}{250,167,12}
\definecolor{yellow}{RGB}{246,250,12}
\definecolor{green}{RGB}{128,238,1}
\definecolor{black}{RGB}{0,0,0}
\definecolor{blue}{RGB}{0,0,255}
\definecolor{red}{RGB}{255,0,0}
\definecolor{sepia}{RGB}{94,38,18}
\newcommand{\ve}[1]{{\rm\bf {#1}}}
\newcommand{\q}[1]{\textcolor{blue}{#1}}
\newcommand{\blue}[1]{\textcolor{blue}{#1}}
\newcommand{\sepia}[1]{\textcolor{sepia}{#1}}
\newcommand{\red}[1]{\textcolor{red}{#1}}
\newcommand{\green}[1]{\textcolor{green}{#1}}
\newcommand{\yellow}[1]{\textcolor{yellow}{#1}}
\newcommand{\orange}[1]{\textcolor{orange}{#1}}
\definecolor{burlywood}{RGB}{255,211,155}
\definecolor{chocolate}{RGB}{255,127,36}
\definecolor{tan}{RGB}{210,180,140}
%
\def\onethird{{\textstyle{1\over3}}}
\def\twothirds{{\textstyle{2\over3}}}
\def\fourthirds{{\textstyle{4\over3}}}
\def\onehalf{{\textstyle{1\over2}}}
\def\threehalfs{{\textstyle{3\over2}}}
%
\newcommand{\pd}{\partial}
\newcommand{\aMLT}{\alpha_{\rm MLT}}
\newcommand{\Fconv}{F_{\rm conv}}
\newcommand{\Frad}{F_{\rm rad}}
\newcommand{\Ftot}{F_{\rm tot}}
\newcommand{\Hp}{H_p}
\newcommand{\prad}{p_{\rm rad}}
\newcommand{\pgas}{p_{\rm gas}}
%
\title{Theoretical Astrophysics I: Physics of Sun and Stars\\
Lecture 5: Convection in stars versus convection simulations}
\author{\texorpdfstring{\sepia{Petri K\"{a}pyl\"{a} Ivan Mili\'{c}}\newline\blue{\url{pkapyla, milic@leibniz-kis.de}}}{}}
\institute{Institut f\"ur Sonnenphysik - KIS, Freiburg}
\date{\today}
%
\begin{document}
\frame{\titlepage}

% Convective conundrum: velocity amplitudes in the Sun vs. simulations
% Influence of rotation on convection
% Differential rotation.
% Driving on convection: local or non-local?
% Effects of magnetic fields?
% Induction equation
% Dynamo problem.
% Large-scale and small-scale dynamos.


\section{Convection}
%
%
\frame{
\frametitle{Radiative Transfer Equation}
\begin{minipage}{0.55\linewidth}
\begin{itemize}
\item Now, it makes sense that the absorption and emission properties of the medium depend on two things:
\item Amount of matter capable of absorbing/emitting
\item The inherent properties of the matter at the given temperature ($T$ is very important!)
\item So we define: 
\begin{align}
\kappa_\nu = \chi_\nu / \rho \\
j_\nu = \eta_\nu / \rho
\end{align}
\item So our equation becomes: 
\begin{equation}
\rho \frac{dI_\nu}{ds} = -\kappa_\nu I_\nu + j_\nu
\end{equation}
\end{itemize}
\end{minipage}
\begin{minipage}{0.44\linewidth}
\begin{figure}
\includegraphics[width=6cm]{figures/rte.jpg}
%\caption*{Solar supergranulation from SOHO MDI.}
\end{figure}
\end{minipage}
}
%
\frame{
\frametitle{Few remarks}
\begin{minipage}{0.55\linewidth}
\begin{itemize}
\begin{equation}
\frac{1}{\rho} \frac{dI_\nu}{ds} = -\kappa_\nu I_\nu + j_\nu
\end{equation}
\item This equation does not involve any new physical laws, it is basically a Boltzmann transport equation. It is a mathematical tool. 
\item $ds$ is a so-called ``ray'', its relationship to $dx, dy, dz$ will depend on the geometry we choose and the context. 
\item We will assume that coefficients of absorption and emission are isotropic, but that they do depend on the frequency / wavelength. 
\item \textbf{However - the amount of absorbed intensity is not isotropic as it depends on the intensity itself}.
\end{itemize}
\end{minipage}
\begin{minipage}{0.44\linewidth}
\begin{figure}
\includegraphics[width=6cm]{figures/rte.jpg}
%\caption*{Solar supergranulation from SOHO MDI.}
\end{figure}
\end{minipage}
}

\frame{
\frametitle{Optical depth and Source function}
\begin{minipage}{0.55\linewidth}
\begin{itemize}
\item As we like dimensionless quantities, we often do the following: 
\begin{equation}
\frac{dI_\nu}{-\rho \kappa_nu ds} = I_\nu - j_\nu / \kappa_\nu
\end{equation}
\item And we get: 
\begin{equation}
\frac{dI_\nu}{-\rho \kappa_nu ds} = I_\nu - j_\nu / (\rho \kappa_\nu)
\end{equation}

\begin{align}
\kappa_\nu = \chi_\nu / \rho \\
j_\nu = \eta_\nu / \rho
\end{align}
\item So our equation becomes: 
\begin{equation}
\rho \frac{dI_\nu}{d\tau_\nu} = I_\nu - S_\nu
\end{equation}
\item A lot of interesting solutions will involve exponents of $\tau_\nu$ - nice that $\tau$ is exponential!
\end{itemize}
\end{minipage}
\begin{minipage}{0.44\linewidth}
\begin{figure}
\includegraphics[width=6cm]{figures/rte.jpg}
%\caption*{Solar supergranulation from SOHO MDI.}
\end{figure}
\end{minipage}
}

\end{document}



