\documentclass[12pt]{article}
\special{papersize=210mm,297mm}
\usepackage[top=1.5cm,bottom=1.5cm,left=2cm,right=2cm]{geometry}
\usepackage[skip=12pt plus1pt, indent=0pt]{parskip}

% Usual packages
\usepackage{graphicx}
\usepackage{hyperref}

\usepackage{fancyhdr}
\usepackage{amsmath}
\usepackage{cancel}
\usepackage{amssymb}
\usepackage{eurosym}
\usepackage{multicol}
\usepackage{rotate}
\usepackage{tabularx}
\usepackage{floatrow}
\usepackage{color}
\usepackage{colortbl}
\usepackage{braket}
\usepackage[shortcuts]{extdash}
\setlength{\headheight}{15.2pt}
\pagestyle{fancy}
\renewcommand{\headrulewidth}{0.5pt}
\renewcommand{\footrulewidth}{0.5pt}

\fancyhf{}
%%%% colors
\definecolor{orange}{RGB}{250,167,12}
\definecolor{yellow}{RGB}{246,250,12}
\definecolor{green}{RGB}{128,238,1}
\definecolor{green2}{RGB}{0,250,154}
\definecolor{black}{RGB}{0,0,0}
\definecolor{blue}{RGB}{0,0,255}
\definecolor{red}{RGB}{255,0,0}
\definecolor{white}{RGB}{255,255,255}
\definecolor{burlywood1}{RGB}{255,211,155}
\definecolor{chocolate1}{RGB}{255,127,36}
\definecolor{sepia}{RGB}{94,38,18}
\newcommand{\blue}[1]{\textcolor{blue}{#1}}
\newcommand{\sepia}[1]{\textcolor{sepia}{#1}}
\newcommand{\red}[1]{\textcolor{red}{#1}}
\newcommand{\green}[1]{\textcolor{green}{#1}}
\newcommand{\yellow}[1]{\textcolor{yellow}{#1}}
\newcommand{\orange}[1]{\textcolor{orange}{#1}}
%%%% mathematical definitions
\def\onethird{{\textstyle{1\over3}}}
\def\onesixth{{\textstyle{1\over6}}}
\DeclareMathOperator{\Tr}{Tr}
\def\npab{\noindent \textbullet ~}
\def\npa{\noindent}
\def\npat{\noindent \textcolor{blue}{$\blacktriangleright$} ~}
\def\npac{\noindent $\circledast$ ~}
\def\nn{{\bf \nabla}}
\def\df{{\rm d}}
\def\cro{\times}
\def\ip{i^{'}}
\def\jp{j^{'}}
\def\kp{k^{'}}
\def\deg{$^{\circ}$}
\def\rhoc{\textcolor{red}{\rho}}
\def\uic{\textcolor{red}{u_i}}
\def\ujc{\textcolor{red}{u_j}}
\def\pgc{\textcolor{red}{P_{\rm g}}}
\newcommand{\ve}[1]{{\rm\bf {#1}}}
\newcommand{\Fconv}{F_{\rm conv}}
\def\dd{{\rm d}}
\def\ez{\ve{e}_z}
\def\ey{\ve{e}_y}
\def\ex{\ve{e}_x}
\def\ezero{\ve{e}_{0}}
\def\eplu{\ve{e}_{+1}}
\def\emin{\ve{e}_{-1}}
\def\definition{:=^{\!\!\!\!\!\!\!\textrm{def}}}
\newcommand{\bg}[1]{{\boldsymbol {#1}}}
\newcommand*\tavg[1]{%
  \hbox{%
    \vbox{%
      \hrule height 0.5pt % The actual bar
      \kern0.5ex%         % Distance between bar and symbol
      \hbox{%
        \kern-0.1em%      % Shortening on the left side
        \ensuremath{#1}%
        \kern-0.1em%      % Shortening on the right side
      }%
    }%
  }%
}
\title{Theoretical Astrophysics: Physics of Sun and Stars\\
Homework 2}
\author{P. K\"{a}pyl\"{a}, I. Mili\'{c}}
\date{\today}
%%%%
\begin{document}
\maketitle

\textbf{Deadline for this homework is \textbf{11}/06 23:59}

{\bf Problem 1:} The solar luminosity is $L_\odot = 3.83\cdot
10^{26}$~W. Assume that all of the energy for this luminosity is
provided by the ${\rm p-p\,1}$ chain, and that neutrinos carry off 3\% of the
energy. How many neutrinos are produced per second? What is the
neutrino flux (i.e.\ the number of neutrinos per second per cm$^2$) at
Earth?

The ${\rm p-p\,1}$ chain is comprised of the reactions:
\begin{eqnarray}
  p + p &\rightarrow& ^2{\rm D} + e^+ + \nu,\ (1.18~{\rm MeV})\nonumber \\
  ^2{\rm D} + p &\rightarrow& ^3{\rm He} + \gamma,\ \ \ \ \ \ (5.49~{\rm MeV}) \nonumber \\
  ^3{\rm He} + ^3{\rm He} &\rightarrow& ^4{\rm He} + 2p.\ \ \ \ \ (12.86~{\rm MeV})
\end{eqnarray}
where the first two reactions need to happen twice for the last
reaction to occur, and where $\nu$ signifies an emitted
neutrino. Energy released (in MeV = $10^6$~eV) is given in brackets
after each reaction.


{\bf Solution:} Bearing in mind that the first two reactions occur
twice each time the chain is completed, $E_{\rm pp} = [2 \times (1.18
  + 5.49) + 12.86]$ = 26.2~MeV of energy is released. An ${\rm eV} =
1.602176634 \cdot 10^{-19}$~J, so $E_{\rm pp} \approx 4.198 \cdot
10^{-12}$~J. Because luminosity is energy release per second (unit: W
= J/s), the number of reactions per second is $n_{\rm reac} =
L_\odot / E_{\rm pp} \approx 9.124 \cdot 10^{37}$. As the first step in
the chain happens twice, each time releasing a neutrino, the number of
neutrinos produced per second is $n_\nu = 2 n_{\rm reac} \approx 1.825
\cdot 10^{38}$.

The radius of the Earth's orbit is $d = 1.5 \cdot 10^{13}$~cm, and the
area of a sphere enclosing this is $A_d = 4\pi d^2 \approx 2.827 \cdot
10^{27}$~cm$^2$. Therefore the number of neutrinos passing per second
per cm$^2$ at the Earth is $n_\nu/A_d \approx 6.45 \cdot 10^{11}$.


{\bf Problem 2:} 

a) Why does convection transport heat radially outward although there
is no net mass flux?

{\bf Solution:} In convecting fluid light warm ($T' > 0$) matter rises
($u_r > 0$) and cool ($T'<0$) dense matter descends ($u_r < 0$), where
$T' = T - \overline{T}$ is the temperature perturbation with respect
to the average temperature $\overline{T}$. The energy flux carried by
convection is proportional to the product $u_r T'$ which is $>0$,
i.e., radially outward, for both upflows and downflows. It is also
essential the that heat is lost near the surface when radiation
becomes efficient.


b) Estimate the superadiabatic temperature gradient (in order of
magnitude fashion) in the Sun, by making use of the mixing length
expression for $\Fconv$:
\begin{equation}
  \Fconv = c_p \rho T g^{1/2} \frac{\ell^2}{4\sqrt{2}H_p^{3/2}} (\nabla-\nabla_{\rm ad})^{3/2}.
\end{equation}
Use the average values of density and temperature of the Sun, and $c_p
= 2.07\cdot10^4$~J~K$^{-1}$~kg$^{-1}$. 

Hint: Recall that $\ell =
\alpha_{\rm MLT}H_p$ and bear in mind the definition of the pressure
scale height from previous homework. When is this result a good
approximation and when not? Why?

{\bf Solution:} Solve first for $\nabla-\nabla_{\rm ad}$:
\begin{eqnarray}
%  \Fconv = c_p \rho T g^{1/2} \frac{\ell^2}{4\sqrt{2}H_p^{3/2}} (\nabla-\nabla_{\rm ad})^{3/2}.
(\nabla-\nabla_{\rm ad})^{3/2} &=&  \frac{4\sqrt{2} H_p^{3/2}}{c_p \rho T g^{1/2} \ell^2} \Fconv,\\
(\nabla-\nabla_{\rm ad})^{3/2} &=&  \frac{4\sqrt{2} H_p^{3/2}}{c_p \rho T g^{1/2} \alpha^2 H_p^2} \Fconv,\\
\Delta\nabla \equiv \nabla-\nabla_{\rm ad} &=& \frac{(4\sqrt{2})^{2/3}}{(c_p \rho T g^{1/2} \alpha^2 H_p^{1/2})^{2/3}} \Fconv^{2/3},
\end{eqnarray}
where $\ell = \alpha H_p$ was used to eliminate $\ell$. Recalling that
\begin{equation}
H_p = \frac{kT}{\mu g} = {\cal R} \frac{T}{g}.
\end{equation}
We further recall that:
\begin{equation}
{\cal R} = c_p - c_V = c_p \left( 1 - \frac{1}{\gamma} \right) = c_p \nabla_{\rm ad}.
\end{equation}
Therefore,
\begin{eqnarray}
\Delta\nabla = \frac{(4\sqrt{2})^{2/3}}{c_p \rho^{2/3} T \alpha^{4/3} \nabla_{\rm ad}^{1/3}} \Fconv^{2/3},
\end{eqnarray}
We use the mean solar density $\overline{\rho} \approx 1.4\cdot
10^3$~kg~m$^{-3}$ and the mean temperature $\overline{T} \approx
6\cdot 10^4 $~K from the lectures and recall that $\nabla_{\rm ad} =
{2 \over 5}$. We approximate the mixing length parameter as
$\alpha=1$. Finally, assuming that the solar flux is transported by
convection, we obtain at $r=R_\odot$ that $\Fconv = 6.29 \cdot
10^7$~W~m$^{-2}$. Substituting all the values we obtain
\begin{equation}
\Delta \nabla \approx 4.4 \cdot 10^{-6}.
\end{equation}
This shows that a very small superadiabaticity is enough to transport
all of the solar luminosity.

This value is close to the one we would expect near the base of the
solar convection zone. This is because there the density and
temperature are large, and the mean density of the Sun is much larger
than the density at the base of the convection zone. However, near the
surface the temperature and especially the density are much smaller
and we would expect $\Delta\nabla$ to be much larger there.



{\bf Problem 3:} Use the equations of hydrostatic equilibrium and mass
conservation:
\begin{eqnarray}
  \frac{dp}{dr} = -\rho \frac{Gm}{r^2},\ \ \frac{dm}{dr} = 4\pi r^2\rho, \label{equ:dpdm}
\end{eqnarray}
assuming a polytropic equation of state
\begin{equation}
p = K \rho^\gamma, \label{equ:ppoly}
\end{equation}
where $\gamma = 1 + \frac{1}{n}$ is the polytropic exponent and $n$ is
the polytropic index, to derive the Lane-Emden equation:
\begin{equation}
\frac{1}{\xi^2} \frac{d}{d\xi} \left(\xi^2 \frac{d\theta}{d\xi} \right) = -\theta^n.\label{equ:LaneEmden}
\end{equation}
Here $\theta$ and $\xi$ are the non-dimensional density and radius
defined as
\begin{equation}
\rho = \rho_{\rm c} \theta^n, \ \ \mbox{and}\ \ r = \alpha\xi, \label{equ:rrho}
\end{equation}
and where $\alpha^2$ equals a constant that arises in the derivation
of Eq.(\ref{equ:LaneEmden}).


{\bf Solution:} Start from equations~(\ref{equ:dpdm}) to find that:
\begin{eqnarray}
\frac{r^2}{\rho} \frac{\dd p}{\dd r} &=& -Gm, \\
\frac{\dd}{\dd r} \left( \frac{r^2}{\rho} \frac{\dd p}{\dd r} \right) &=& -G \frac{\dd m}{\dd r} = -4\pi G \rho r^2, \\
\frac{1}{r^2 }\frac{\dd}{\dd r} \left( \frac{r^2}{\rho} \frac{\dd p}{\dd r} \right) &=& -4\pi G \rho.\label{equ:poisson}
\end{eqnarray}
Make use of Eqs.(\ref{equ:ppoly}) and (\ref{equ:rrho}) to write out
the pressure and its gradient:
\begin{eqnarray}
p &=& K \rho^{1+\frac{1}{n}} = K \rho_c^{1+\frac{1}{n}}\theta^{n+1}, \\
\frac{\dd p}{\dd r} = \frac{\dd p}{\alpha \dd\xi} &=& K \rho_c^{1+\frac{1}{n}} (n+1) \theta^n \frac{\dd\theta}{\alpha\dd\xi},
\end{eqnarray}
where we used $\dd r = \alpha\dd\xi$. Now we can further see that
\begin{eqnarray}
\frac{r^2}{\rho} \frac{\dd p}{\dd r} = \frac{\alpha^2 \xi^2}{\rho_c \theta^n} K \rho_c^{1+\frac{1}{n}} (n+1) \theta^n \frac{\dd\theta}{\alpha\dd\xi} = \alpha K (n+1) \rho_c^{\frac{1}{n}}\xi^2 \frac{\dd\theta}{\dd\xi}.
\end{eqnarray}
Now we are in a position to write Eq.~(\ref{equ:poisson}) with the new
variables:
\begin{eqnarray}
  \frac{1}{\alpha^2 \xi^2} \frac{1}{\alpha} \frac{\dd}{\dd\xi} \left[ \alpha K(n+1)\rho_c^{\frac{1}{n}} \xi^2 \frac{\dd\theta}{\dd\xi} \right] &=& - 4\pi G \rho_c \theta^n,\\
  \frac{K(n+1)}{4\pi G \alpha ^2} \rho_c^{\frac{1}{n}-1} \frac{1}{\xi^2} \frac{\dd}{\dd\xi}\left(\xi^2 \frac{\dd\theta}{\dd\xi} \right) &=& -\theta^n.
\end{eqnarray}
With the choice:
\begin{equation}
\alpha^2 = \frac{K(n+1)}{4\pi G} \rho_c^{\frac{1}{n}-1}
\end{equation}
we recover the Lane-Emden equation:
\begin{eqnarray}
 \frac{1}{\xi^2} \frac{\dd}{\dd\xi}\left(\xi^2 \frac{\dd\theta}{\dd\xi} \right) &=& -\theta^n.
\end{eqnarray}



The boundary conditions at $\xi=0$ for the Lane-Emden equation are:
\begin{equation}
\theta = 1,\ \mbox{and}\ \frac{d\theta}{d\xi} = 0.
\end{equation}
What do these boundary conditions correspond to?

{\bf Solution:} The first condition simply states that the density is
non-dimensionalized by the central density, i.e.,
\begin{equation}
\theta = \left(\frac{\rho}{\rho_c}\right)^{\frac{1}{n}}.
\end{equation}
The latter condition indicates that there is no density or pressure
gradients at $r=\xi=0$. This means that the gravity must also vanish
at $r=\xi=0$.


Solve the Lane-Emden equation for $n=0,1$, and $5$. For these values
the equation can be solved \emph{analytically}.

{\bf Solution:} Case $n=0$: The Lane-Emden equation reduces to:
\begin{eqnarray}
 \frac{1}{\xi^2} \frac{\dd}{\dd\xi}\left(\xi^2 \frac{\dd\theta}{\dd\xi} \right) = -1.
\end{eqnarray}
Integrating once gives:
\begin{eqnarray}
 \xi^2 \frac{\dd\theta}{\dd\xi} = -\onethird \xi^3 - C.
\end{eqnarray}
Second integration gives:
\begin{eqnarray}
 \theta = D - \frac{C}{\xi} - \onesixth \xi^2.
\end{eqnarray}
We want a finite solution at $\xi=0$ and therefore we need to have
$C=0$. Furthermore, the condition $\theta=1$ at $\xi$ requires that $D
=1$. Therefore the solution for $n=0$ reads
\begin{eqnarray}
 \theta_0 = 1 - \onesixth \xi^2.
\end{eqnarray}

\noindent
Case $n=1$: Make first a transformation:
\begin{equation}
  \theta = \frac{\chi}{\xi},\label{equ:thetat}
\end{equation}
so that the Lane-Emden equation reduces to:
\begin{equation}
\frac{\dd^2 \chi}{\dd \xi^2} = - \frac{\chi^n}{\xi^{n-1}}.
\end{equation}
For $n=1$ this reduces further to:
\begin{equation}
\frac{\dd^2 \chi}{\dd \xi^2} = - \chi.
\end{equation}
General solution of this equation is
\begin{equation}
\chi = C \sin(\xi - \delta),
\end{equation}
where $C$ and $\delta$ are integration constants. From
Eq.~(\ref{equ:thetat}) we get
\begin{equation}
\theta = \frac{C \sin(\xi - \delta)}{\xi}.
\end{equation}
If $\delta\neq 0$ we again have a singularity at $\xi=0$ and therefore
we put $\delta=0$. Furthermore, the boundary condition $\theta=1$ at
$\xi=0$ stipulates that $C=1$ because at the limit of small $\xi$,
$\sin\xi/\xi \longrightarrow 1$. Therefore,
\begin{equation}
\theta_1 = \frac{\sin\xi}{\xi}.
\end{equation}

\noindent
Case $n=5$: We simply state the final result here:
\begin{equation}
\theta_5 = \frac{1}{(1+\onethird \xi^2)^{1/2}}.
\end{equation}
Detailed derivation can be found, e.g., in Chandrasekhar, S. (1939):
``An Introduction to the Study of Stellar Structure'' (Dover
Publications), p.\ 93.


Solve the Lane-Emden equation for $n=1.5$ and $n=3$, and compare the
resuls with the ones found in the textbook by D.\ Prialnik. For this
you will have to solve the equation \emph{numerically}.

{\bf Solution:} See script in the \texttt{homework/python} folder.


{\bf Useful physical constants}
\begin{itemize}
  \item $R_{\odot} = 696 \times 10^6\,{\rm m}$
  \item $M_{\odot} = 1.989 \times 10^{30}\,{\rm kg}$
  \item $L_{\odot} = 3.83 \times 10^{26}$~W
  \item $T^{\rm eff}_{\odot} = 5777\,{\rm K}$
  \item $1\,{\rm AU} = 1.496 \times 10^8\,{\rm km}$
  \item $c = 2.997 \times 10^8\,{\rm m/s}$
  \item $G = 6.674 \times 10^{-11}$~Nm$^2$/kg$^2$
  \item $k = 1.38\cdot10^{-23}$~J/K
  \item $m_{\rm H} = 1.67\cdot10^{-27}$~kg
  \item $h=6.626 \times 10^{-34}$~J~s.
  \item $k=1.38 \times 10^{-23}$~J/K.
\end{itemize}
\end{document}
