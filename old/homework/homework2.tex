\documentclass[12pt]{article}
\special{papersize=210mm,297mm}
\usepackage[top=1.5cm,bottom=1.5cm,left=2cm,right=2cm]{geometry}
\usepackage[skip=12pt plus1pt, indent=0pt]{parskip}

% Usual packages
\usepackage{graphicx}
\usepackage{hyperref}

\usepackage{fancyhdr}
\usepackage{amsmath}
\usepackage{cancel}
\usepackage{amssymb}
\usepackage{eurosym}
\usepackage{multicol}
\usepackage{rotate}
\usepackage{tabularx}
\usepackage{floatrow}
\usepackage{color}
\usepackage{colortbl}
\usepackage{braket}
\usepackage[shortcuts]{extdash}
\setlength{\headheight}{15.2pt}
\pagestyle{fancy}
\renewcommand{\headrulewidth}{0.5pt}
\renewcommand{\footrulewidth}{0.5pt}

\fancyhf{}
%%%% colors
\definecolor{orange}{RGB}{250,167,12}
\definecolor{yellow}{RGB}{246,250,12}
\definecolor{green}{RGB}{128,238,1}
\definecolor{green2}{RGB}{0,250,154}
\definecolor{black}{RGB}{0,0,0}
\definecolor{blue}{RGB}{0,0,255}
\definecolor{red}{RGB}{255,0,0}
\definecolor{white}{RGB}{255,255,255}
\definecolor{burlywood1}{RGB}{255,211,155}
\definecolor{chocolate1}{RGB}{255,127,36}
\definecolor{sepia}{RGB}{94,38,18}
\newcommand{\blue}[1]{\textcolor{blue}{#1}}
\newcommand{\sepia}[1]{\textcolor{sepia}{#1}}
\newcommand{\red}[1]{\textcolor{red}{#1}}
\newcommand{\green}[1]{\textcolor{green}{#1}}
\newcommand{\yellow}[1]{\textcolor{yellow}{#1}}
\newcommand{\orange}[1]{\textcolor{orange}{#1}}
%%%% mathematical definitions
\DeclareMathOperator{\Tr}{Tr}
\def\npab{\noindent \textbullet ~}
\def\npa{\noindent}
\def\npat{\noindent \textcolor{blue}{$\blacktriangleright$} ~}
\def\npac{\noindent $\circledast$ ~}
\def\nn{{\bf \nabla}}
\def\df{{\rm d}}
\def\cro{\times}
\def\ip{i^{'}}
\def\jp{j^{'}}
\def\kp{k^{'}}
\def\deg{$^{\circ}$}
\def\rhoc{\textcolor{red}{\rho}}
\def\uic{\textcolor{red}{u_i}}
\def\ujc{\textcolor{red}{u_j}}
\def\pgc{\textcolor{red}{P_{\rm g}}}
\newcommand{\ve}[1]{{\rm\bf {#1}}}
\newcommand{\Fconv}{F_{\rm conv}}
\def\ez{\ve{e}_z}
\def\ey{\ve{e}_y}
\def\ex{\ve{e}_x}
\def\ezero{\ve{e}_{0}}
\def\eplu{\ve{e}_{+1}}
\def\emin{\ve{e}_{-1}}
\def\definition{:=^{\!\!\!\!\!\!\!\textrm{def}}}
\newcommand{\bg}[1]{{\boldsymbol {#1}}}
\newcommand*\tavg[1]{%
  \hbox{%
    \vbox{%
      \hrule height 0.5pt % The actual bar
      \kern0.5ex%         % Distance between bar and symbol
      \hbox{%
        \kern-0.1em%      % Shortening on the left side
        \ensuremath{#1}%
        \kern-0.1em%      % Shortening on the right side
      }%
    }%
  }%
}
\title{Theoretical Astrophysics: Physics of Sun and Stars\\
Homework 2}
\author{P. K\"{a}pyl\"{a}, I. Mili\'{c}}
\date{\today}
%%%%
\begin{document}
\maketitle

\textbf{Deadline for this homework is \textbf{11}/06 23:59}

{\bf Problem 1:} The solar luminosity is $L_\odot = 3.83\cdot
10^{26}$~W. Assume that all of the energy for this luminosity is
provided by the ${\rm p-p\,1}$ chain, and that neutrinos carry off 3\% of the
energy. How many neutrinos are produced per second? What is the
neutrino flux (i.e.\ the number of neutrinos per second per cm$^2$) at
Earth?

The ${\rm p-p\,1}$ chain is comprised of the reactions:
\begin{eqnarray}
  p + p &\rightarrow& ^2{\rm D} + e^+ + \nu,\ (1.18~{\rm MeV})\nonumber \\
  ^2{\rm D} + p &\rightarrow& ^3{\rm He} + \gamma,\ \ \ \ \ \ (5.49~{\rm MeV}) \nonumber \\
  ^3{\rm He} + ^3{\rm He} &\rightarrow& ^4{\rm He} + 2p.\ \ \ \ \ (12.86~{\rm MeV})
\end{eqnarray}
where the first two reactions need to happen twice for the last
reaction to occur, and where $\nu$ signifies an emitted
neutrino. Energy released (in MeV = $10^6$~eV) is given in brackets
after each reaction.


{\bf Problem 2:} 

a) Why does convection transport heat radially outward although there
is no net mass flux?

b) Estimate the superadiabatic temperature gradient (in order of
magnitude fashion) in the Sun, by making use of the mixing length
expression for $\Fconv$:
\begin{equation}
  \Fconv = c_p \rho T g^{1/2} \frac{\ell^2}{4\sqrt{2}H_p^{3/2}} (\nabla-\nabla_{\rm ad})^{3/2}.
\end{equation}
Use the average values of density and temperature of the Sun, and $c_p
= 2.07\cdot10^4$~J~K$^{-1}$~kg$^{-1}$. 

Hint: Recall that $\ell =
\alpha_{\rm MLT}H_p$ and bear in mind the definition of the pressure
scale height from previous homework. When is this result a good
approximation and when not? Why?


{\bf Problem 3:} Use the equations of hydrostatic equilibrium and mass
conservation:
\begin{eqnarray}
  \frac{dp}{dr} = -\rho \frac{Gm}{r^2},\ \ \frac{dm}{dr} = 4\pi r^2\rho, 
\end{eqnarray}
assuming a polytropic equation of state
\begin{equation}
p = K \rho^\gamma,
\end{equation}
where $\gamma = 1 + \frac{1}{n}$ is the polytropic exponent and $n$ is
the polytropic index, to derive the Lane-Emden equation:
\begin{equation}
\frac{1}{\xi^2} \frac{d}{d\xi} \left(\xi^2 \frac{d\theta}{d\xi} \right) = -\theta^n.\label{equ:LaneEmden}
\end{equation}
Here $\theta$ and $\xi$ are the non-dimensional density and radius
defined as
\begin{equation}
\rho = \rho_{\rm c} \theta^n, \ \ \mbox{and}\ \ r = \alpha\xi,
\end{equation}
and where $\alpha^2$ equals a constant that arises in the derivation
of Eq.(\ref{equ:LaneEmden}).

The boundary conditions at $\xi=0$ for the Lane-Emden equation are:
\begin{equation}
\theta = 1,\ \mbox{and}\ \frac{d\theta}{d\xi} = 0.
\end{equation}
%Explain why we are choosing these specific boundary conditions. (As a
%question for students).
What do these boundary conditions correspond to?

Solve the Lane-Emden equation for $n=0,1$, and $5$. For these values
the equation can be solved \emph{analytically}.

{Solve the Lane-Emden equation for $n=1.5$ and $n=3$, and compare the
  resuls with the ones found in the textbook by D.\ Prialnik. For this
  you will have to solve the equation \emph{numerically}.

{\bf Useful physical constants}
\begin{itemize}
  \item $R_{\odot} = 696 \times 10^6\,{\rm m}$
  \item $M_{\odot} = 1.989 \times 10^{30}\,{\rm kg}$
  \item $L_{\odot} = 3.83 \times 10^{26}$~W
  \item $T^{\rm eff}_{\odot} = 5777\,{\rm K}$
  \item $1\,{\rm AU} = 1.496 \times 10^8\,{\rm km}$
  \item $c = 2.997 \times 10^8\,{\rm m/s}$
  \item $G = 6.674 \times 10^{-11}$~Nm$^2$/kg$^2$
  \item $k = 1.38\cdot10^{-23}$~J/K
  \item $m_{\rm H} = 1.67\cdot10^{-27}$~kg
  \item $h=6.626 \times 10^{-34}$~J~s.
  \item $k=1.38 \times 10^{-23}$~J/K.
\end{itemize}
\end{document}
