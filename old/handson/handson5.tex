\documentclass[12pt]{article}
\special{papersize=210mm,297mm}
\usepackage[top=1.5cm,bottom=1.5cm,left=2cm,right=2cm]{geometry}
\usepackage[skip=12pt plus1pt, indent=0pt]{parskip}

\usepackage{fancyhdr}
\usepackage{graphicx}
\usepackage{amsmath}
\usepackage{cancel}
\usepackage{amssymb}
\usepackage{eurosym}
\usepackage{multicol}
\usepackage{rotate}
\usepackage{tabularx}
\usepackage{floatrow}
\usepackage{color}
\usepackage{colortbl}
\usepackage{braket}
\usepackage{bm}
\usepackage[shortcuts]{extdash}
\setlength{\headheight}{15.2pt}
\pagestyle{fancy}
\renewcommand{\headrulewidth}{0.5pt}
\renewcommand{\footrulewidth}{0.5pt}

\fancyhf{}
%%%% colors
\definecolor{orange}{RGB}{250,167,12}
\definecolor{yellow}{RGB}{246,250,12}
\definecolor{green}{RGB}{128,238,1}
\definecolor{green2}{RGB}{0,250,154}
\definecolor{black}{RGB}{0,0,0}
\definecolor{blue}{RGB}{0,0,255}
\definecolor{red}{RGB}{255,0,0}
\definecolor{white}{RGB}{255,255,255}
\definecolor{burlywood1}{RGB}{255,211,155}
\definecolor{chocolate1}{RGB}{255,127,36}
\definecolor{sepia}{RGB}{94,38,18}
\newcommand{\blue}[1]{\textcolor{blue}{#1}}
\newcommand{\sepia}[1]{\textcolor{sepia}{#1}}
\newcommand{\red}[1]{\textcolor{red}{#1}}
\newcommand{\green}[1]{\textcolor{green}{#1}}
\newcommand{\yellow}[1]{\textcolor{yellow}{#1}}
\newcommand{\orange}[1]{\textcolor{orange}{#1}}
%%%% mathematical definitions
\DeclareMathOperator{\Tr}{Tr}
\def\pd{\partial}
\def\npab{\noindent \textbullet ~}
\def\npa{\noindent}
\def\npat{\noindent \textcolor{blue}{$\blacktriangleright$} ~}
\def\npac{\noindent $\circledast$ ~}
\def\nn{{\bf \nabla}}
\def\df{{\rm d}}
\def\cro{\times}
\def\ip{i^{'}}
\def\jp{j^{'}}
\def\kp{k^{'}}
\def\deg{$^{\circ}$}
\def\rhoc{\textcolor{red}{\rho}}
\def\uic{\textcolor{red}{u_i}}
\def\ujc{\textcolor{red}{u_j}}
\def\pgc{\textcolor{red}{P_{\rm g}}}
\newcommand{\ve}[1]{{\rm\bf {#1}}}
\def\ez{\ve{e}_z}
\def\ey{\ve{e}_y}
\def\ex{\ve{e}_x}
\def\ezero{\ve{e}_{0}}
\def\eplu{\ve{e}_{+1}}
\def\emin{\ve{e}_{-1}}
\def\definition{:=^{\!\!\!\!\!\!\!\textrm{def}}}
\newcommand{\bg}[1]{{\boldsymbol {#1}}}
\newcommand*\tavg[1]{%
  \hbox{%
    \vbox{%
      \hrule height 0.5pt % The actual bar
      \kern0.5ex%         % Distance between bar and symbol
      \hbox{%
        \kern-0.1em%      % Shortening on the left side
        \ensuremath{#1}%
        \kern-0.1em%      % Shortening on the right side
      }%
    }%
  }%
}
\title{Hands-on exercises 5: Non-dimensionalisation and limits of numerical simulations}
\author{P. K\"{a}pyl\"{a}, I. Mili\'{c}}
\date{\today}
%%%%
\begin{document}
\maketitle

We will take a look at non-dimensionalisation of some of the MHD
equations and discuss the limitations of numerical simulations of
stellar convection zones.

{\bf Problem 1:} Take the dimensional Navier-Stokes equation in the
Boussinesq approximation, where density differences are ignored
everywhere except in the buoyancy term
\begin{equation}
\frac{\pd {\bm u}}{\pd t} + {\bm u}\bm\cdot\bm\nabla {\bm u} = -\bm\nabla\varpi + \nu \nabla^2 {\bm u} + \alpha g T \hat{\bm z},\label{equ:NSbou}
\end{equation}
where $\varpi \sim p/\rho$ is a reduced pressure, $\alpha$ is the
coefficient of thermal expansion, and $\hat{\bm z}$ is the unit vector
in the $z$ direction. Write the equation in non-dimensional form
using, e.g., $x = \ell_{\rm c}\tilde{x}$, $t = \tau_{\rm c}\tilde{t}$,
etc., and choose $\ell_{\rm c}$, $\tau_{\rm c}$, and other units such
that the non-dimensional version of Eq.~(\ref{equ:NSbou}) contains
system parameters ${\rm Ma}^2 = u_c^2/(p_c/\rho_c)$, ${\rm Pr} =
\nu/\chi$, and ${\rm Ra} = g \alpha T_c L^3/(\nu\chi)$.


{\bf Problem 2:} Estimate the Kolmogorov scale $\ell_\nu$ (scale at
which kinetic energy is dissipated to heat) by assuming that the
energy transfer rate from large to small scales equals the kinetic
energy dissipation rate.

Calculate the ratio $\ell_\nu/L$ where $L$ is the integral (system)
scale and relate this to the number of grid points you need in a
simulation to fully capture the dynamics (aka a \emph{direct
simulation}). Bear in mind the definition of the Reynolds number
\begin{equation}
{\rm Re} = \frac{u\ell}{\nu}.
\end{equation}
For the solar convection zone ${\rm Re} \sim 10^{12}$, where $L =
0.3R_\odot$. How many grid points would you need to fully capture
this? A typical current simulation has $1000^3$ grid points and use
around $10^3$ CPU cores. How much more would be needed for a direct
simulation of the solar convection zone?

The timestep of the simulations is restricted by the
Courant-Friedrichs-Levy (CFL) condition that states that
\begin{equation}
  \delta t \leq C_{\rm CFL} \frac{\Delta x}{u_{\rm s}},
\end{equation}
where $C_{\rm CFL}$ is a constant of the order of unity, $\Delta x$ is
the grid spacing, and $u_{\rm s}$ is the fastest signal propagation
speed. Estimate the timestep for a direct simulation of the solar
convection zone if the fastest signal is the sound speed at the base
($c_{\rm s} \sim 200$~km~s$^{-1}$) or the maximum convective velocity
($u_{\rm conv} \sim 2$~km~s$^{-1}$).

Computing a single timestep in the typical simulations mentioned above
takes about $0.1$~s in wall-clock time. Estimate the time-to-solution
for a direct simulation of the solar convection zone over a sunspot
cycle (11~years) based on the timestep estimates computed
earlier. What can you conclude?


%\newpage
{\bf Useful physical constants}
\begin{itemize}
  \item $R_{\odot} = 696 \times 10^6\,{\rm m}$
  \item $M_{\odot} = 1.989 \times 10^{30}\,{\rm kg}$
  \item $L_{\odot} = 3.83 \times 10^{26}$~W
  \item $T^{\rm eff}_{\odot} = 5777\,{\rm K}$
  \item $1\,{\rm AU} = 1.496 \times 10^8\,{\rm km}$
  \item $c = 2.997 \times 10^8\,{\rm m/s}$
  \item $G = 6.674 \times 10^{-11}$~Nm$^2$/kg$^2$
  \item $k = 1.38\cdot10^{-23}$~J/K
  \item $m_{\rm H} = 1.67\cdot10^{-27}$~kg
\end{itemize}









\end{document}
