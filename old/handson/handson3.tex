\documentclass[12pt]{article}
\special{papersize=210mm,297mm}
\usepackage[top=1.5cm,bottom=1.5cm,left=2cm,right=2cm]{geometry}
\usepackage[skip=12pt plus1pt, indent=0pt]{parskip}

\usepackage{fancyhdr}
\usepackage{graphicx}
\usepackage{amsmath}
\usepackage{cancel}
\usepackage{amssymb}
\usepackage{eurosym}
\usepackage{multicol}
\usepackage{rotate}
\usepackage{tabularx}
\usepackage{floatrow}
\usepackage{color}
\usepackage{colortbl}
\usepackage{braket}
\usepackage[shortcuts]{extdash}
\setlength{\headheight}{15.2pt}
\pagestyle{fancy}
\renewcommand{\headrulewidth}{0.5pt}
\renewcommand{\footrulewidth}{0.5pt}

\fancyhf{}
%%%% colors
\definecolor{orange}{RGB}{250,167,12}
\definecolor{yellow}{RGB}{246,250,12}
\definecolor{green}{RGB}{128,238,1}
\definecolor{green2}{RGB}{0,250,154}
\definecolor{black}{RGB}{0,0,0}
\definecolor{blue}{RGB}{0,0,255}
\definecolor{red}{RGB}{255,0,0}
\definecolor{white}{RGB}{255,255,255}
\definecolor{burlywood1}{RGB}{255,211,155}
\definecolor{chocolate1}{RGB}{255,127,36}
\definecolor{sepia}{RGB}{94,38,18}
\newcommand{\blue}[1]{\textcolor{blue}{#1}}
\newcommand{\sepia}[1]{\textcolor{sepia}{#1}}
\newcommand{\red}[1]{\textcolor{red}{#1}}
\newcommand{\green}[1]{\textcolor{green}{#1}}
\newcommand{\yellow}[1]{\textcolor{yellow}{#1}}
\newcommand{\orange}[1]{\textcolor{orange}{#1}}
%%%% mathematical definitions
\DeclareMathOperator{\Tr}{Tr}
\def\npab{\noindent \textbullet ~}
\def\npa{\noindent}
\def\npat{\noindent \textcolor{blue}{$\blacktriangleright$} ~}
\def\npac{\noindent $\circledast$ ~}
\def\nn{{\bf \nabla}}
\def\df{{\rm d}}
\def\cro{\times}
\def\ip{i^{'}}
\def\jp{j^{'}}
\def\kp{k^{'}}
\def\deg{$^{\circ}$}
\def\rhoc{\textcolor{red}{\rho}}
\def\uic{\textcolor{red}{u_i}}
\def\ujc{\textcolor{red}{u_j}}
\def\pgc{\textcolor{red}{P_{\rm g}}}
\newcommand{\ve}[1]{{\rm\bf {#1}}}
\def\ez{\ve{e}_z}
\def\ey{\ve{e}_y}
\def\ex{\ve{e}_x}
\def\ezero{\ve{e}_{0}}
\def\eplu{\ve{e}_{+1}}
\def\emin{\ve{e}_{-1}}
\def\definition{:=^{\!\!\!\!\!\!\!\textrm{def}}}
\newcommand{\bg}[1]{{\boldsymbol {#1}}}
\newcommand*\tavg[1]{%
  \hbox{%
    \vbox{%
      \hrule height 0.5pt % The actual bar
      \kern0.5ex%         % Distance between bar and symbol
      \hbox{%
        \kern-0.1em%      % Shortening on the left side
        \ensuremath{#1}%
        \kern-0.1em%      % Shortening on the right side
      }%
    }%
  }%
}
\title{Hands-on exercises 3: Physics of Gas and radiation}
\author{P. K\"{a}pyl\"{a}, I. Mili\'{c}}
\date{\today}
%%%%
\begin{document}
\maketitle


These are mostly analytical exercises. But, as we deal with the real objects, sometimes we will want to calculate the physical values of our results. For that, it is very convenient to start a simple python session. We will also encourage you to do so :-) 

{\bf Problem 1:} For isotropic radiation, find the relationship between the energy density of the radiation gas (i.e. photons) and the ``radiation pressure''. I.e. the pressure the photons would exert if they behaved like a gas. (You are supposed to get: $u = 3p$).  

{\bf Problem 2:} Show that the radiation pressure integrated over all wavelengths/frequencies, for a blackbody, should be $p = \frac{1}{3} \frac{4 \sigma}{c} T^4$.  

{\bf Problem 3:} Convince yourself that, for stellar atmospheres, the mean molecular mass depends on the degree of ionization. For given gas pressure, assuming the gas consists only of hydrogen, calculate mass density for molecular, completely neutral and completely ionized hydrogen. 

{\bf Problem 4:} Something about adiabatic exponent. 

{\bf Problem 5:} Derive the expressions for rate of change of element abundance (the funny one with deltas from the last class):
\begin{equation}

\end{equation} 

{\bf Problem 6:} Let's assume that the energy flux ($i.e. power, luminosity$, like in the book) in the star is proportional to the gradient of the temperature. Write down the four equations of the stellar structure that govern the stellar structure. 

{\bf Problem 7:} Convince yourself that, if the pressure only depends on the density and not on temperature, we can solve the pressure and density stratification of the star separately from the flux transport. For the dependence of the type: 
\begin{equation}
p = K \rho^\gamma,
\end{equation}  
formulate the so-called \emph{Lane-Emden equation}. Solving this equation will be the centerpiece of our next lecture and your next homework ;-) 

{\bf Useful physical constants}
\begin{itemize}
  \item $R_{\odot} = 696 \times 10^6\,{\rm m}$
  \item $M_{\odot} = 1.989 \times 10^{30}\,{\rm kg}$
  \item $L_{\odot} = 3.83 \times 10^{26}$~W
  \item $T^{\rm eff}_{\odot} = 5777\,{\rm K}$
  \item $1\,{\rm AU} = 1.496 \times 10^8\,{\rm km}$
  \item $c = 2.997 \times 10^8\,{\rm m/s}$
  \item $G = 6.674 \times 10^{-11}$~Nm$^2$/kg$^2$
  \item $k = 1.38\cdot10^{-23}$~J/K
  \item $m_{\rm H} = 1.67\cdot10^{-27}$~kg
\end{itemize}









\end{document}
