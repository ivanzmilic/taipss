\documentclass[12pt]{article}
\special{papersize=210mm,297mm}
\usepackage[top=1.5cm,bottom=1.5cm,left=2cm,right=2cm]{geometry}
\usepackage[skip=12pt plus1pt, indent=0pt]{parskip}

\usepackage{fancyhdr}
\usepackage{graphicx}
\usepackage{amsmath}
\usepackage{cancel}
\usepackage{amssymb}
\usepackage{eurosym}
\usepackage{multicol}
\usepackage{rotate}
\usepackage{tabularx}
\usepackage{floatrow}
\usepackage{color}
\usepackage{colortbl}
\usepackage{braket}
\usepackage{bm}
\usepackage[shortcuts]{extdash}
\setlength{\headheight}{15.2pt}
\pagestyle{fancy}
\renewcommand{\headrulewidth}{0.5pt}
\renewcommand{\footrulewidth}{0.5pt}

\fancyhf{}
%%%% colors
\definecolor{orange}{RGB}{250,167,12}
\definecolor{yellow}{RGB}{246,250,12}
\definecolor{green}{RGB}{128,238,1}
\definecolor{green2}{RGB}{0,250,154}
\definecolor{black}{RGB}{0,0,0}
\definecolor{blue}{RGB}{0,0,255}
\definecolor{red}{RGB}{255,0,0}
\definecolor{white}{RGB}{255,255,255}
\definecolor{burlywood1}{RGB}{255,211,155}
\definecolor{chocolate1}{RGB}{255,127,36}
\definecolor{sepia}{RGB}{94,38,18}
\newcommand{\blue}[1]{\textcolor{blue}{#1}}
\newcommand{\sepia}[1]{\textcolor{sepia}{#1}}
\newcommand{\red}[1]{\textcolor{red}{#1}}
\newcommand{\green}[1]{\textcolor{green}{#1}}
\newcommand{\yellow}[1]{\textcolor{yellow}{#1}}
\newcommand{\orange}[1]{\textcolor{orange}{#1}}
%%%% mathematical definitions
\DeclareMathOperator{\Tr}{Tr}
\def\pd{\partial}
\def\npab{\noindent \textbullet ~}
\def\npa{\noindent}
\def\npat{\noindent \textcolor{blue}{$\blacktriangleright$} ~}
\def\npac{\noindent $\circledast$ ~}
\def\nn{{\bf \nabla}}
\def\df{{\rm d}}
\def\cro{\times}
\def\ip{i^{'}}
\def\jp{j^{'}}
\def\kp{k^{'}}
\def\deg{$^{\circ}$}
\def\rhoc{\textcolor{red}{\rho}}
\def\uic{\textcolor{red}{u_i}}
\def\ujc{\textcolor{red}{u_j}}
\def\pgc{\textcolor{red}{P_{\rm g}}}
\newcommand{\ve}[1]{{\rm\bf {#1}}}
\def\ez{\ve{e}_z}
\def\ey{\ve{e}_y}
\def\ex{\ve{e}_x}
\def\ezero{\ve{e}_{0}}
\def\eplu{\ve{e}_{+1}}
\def\emin{\ve{e}_{-1}}
\def\definition{:=^{\!\!\!\!\!\!\!\textrm{def}}}
\def\mue{\mu_{\rm e}}
\def\xis{\xi_{\rm s}}
\def\mH{m_{\rm H}}
\def\ppc{p_{\rm c}}
\def\TTc{T_{\rm c}}
\def\rhoc{\rho_{\rm c}}
\def\pmom{p_{\rm m}}
\def\pgas{p_{\rm gas}}
\def\pedeg{p_{\rm e,deg}}
\def\perdeg{p_{\rm e,r-deg}}
\def\pmomz{p_{\rm m0}}
\def\onethird{{\textstyle{1\over3}}}
\def\fourthirds{{\textstyle{4\over3}}}
\def\threehalfs{{\textstyle{3\over2}}}
\newcommand{\bg}[1]{{\boldsymbol {#1}}}
\newcommand*\tavg[1]{%
  \hbox{%
    \vbox{%
      \hrule height 0.5pt % The actual bar
      \kern0.5ex%         % Distance between bar and symbol
      \hbox{%
        \kern-0.1em%      % Shortening on the left side
        \ensuremath{#1}%
        \kern-0.1em%      % Shortening on the right side
      }%
    }%
  }%
}
\title{Hands-on exercises 7: Degenerate electron gas, properties of white dwarfs, and thermonuclear instability}
\author{P. K\"{a}pyl\"{a}, I. Mili\'{c}}
\date{\today}
%%%%
\begin{document}
\maketitle

{\bf Problem 1:} Use the Heisenberg uncertainity principle and Pauli
exclusion principle to derive the equations of state for
non-relativistic and relativistic degenerate electron gas:
\begin{equation}
p_{\rm e,deg} = \frac{h^2}{20 m_{\rm e}} \left(\frac{3}{\pi} \right)^{2/3} \frac{1}{m_{\rm H}^{5/3}} \left( \frac{\rho}{\mu_{\rm e}} \right)^{5/3},
\end{equation}
and 
\begin{equation}
p_{\rm e,r-deg} = \frac{hc}{8} \left(\frac{3}{\pi} \right)^{1/3} \frac{1}{m_{\rm H}^{4/3}} \left( \frac{\rho}{\mu_{\rm e}} \right)^{4/3},
\end{equation}
where $h = 6.626\cdot 10^{-34}$J~Hz$^{-1}$ is the Planck constant,
$m_{\rm e}$ is the mass of the electron, $m_{\rm H}$ is the atomic
mass unit, and $\mu_{\rm e}^{-1}$ is the average number of free
electron per nucleon.

Compare the degenerate electron pressure and the the gas pressure from
ideal gas equation at the centre of the Sun with solar composition of
$\mu=0.62$ and $\mu_{\rm e} = 1.17$. Use $\rho_{\rm c} = 1.6\cdot
10^5$~kg~m$^{-3}$ and $T_{\rm c} = 1.57\cdot 10^7$~K for the central
density and temperature of the Sun.


{\bf Solution:} The Heisenberg uncertainity principle states that the
position and momentum ($\pmom$) of a particle cannot be determined
arbitrarily accurately. Quantitatively this is given by:
\begin{equation}
\Delta V \Delta^3 \pmom \geq h^3,\label{equ:Heisenberg}
\end{equation}
where $h$ is the Planck constant. On the other hand, according to the
Pauli exclusion principle no two particles can occupy the same quantum
state (same momentum and spin). Therefore, in a degenerate electron
gas an element in phase space (location and momentum) can be occupied
by two electrons (spin up and down). In a state of complete
degeneration all the quantum states up to a maximum momentum value and
Eq.~(\ref{equ:Heisenberg}) becomes equality. Then the number of
electrons with momenta between $(\pmom,\pmom+d\pmom)$ per unit volume:
\begin{equation}
n_{\rm e}(\pmom)d\pmom = \frac{2}{\Delta V} = \frac{2}{h^3} 4\pi \pmom^2 d\pmom.\label{equ:ne}
\end{equation}
Integrating this equation yields
\begin{eqnarray}
\int_0^{\pmomz} n_{\rm e}(\pmom)d\pmom &=& \int_0^{\pmomz} \frac{2}{h^3} 4\pi \pmom^2 d\pmom,\\
n_{\rm e} = \frac{8\pi}{3 h^3} \pmomz^3,\ \ &\mbox{or}& \ \ \pmomz = \left( \frac{3n_{\rm e}h^3}{8\pi} \right)^{1/3}.\label{equ:momz}
\end{eqnarray}
Now we use the pressure integral from the lectures:
\begin{eqnarray}
p = \onethird \int u \pmom n(\pmom) d\pmom,\label{equ:pintegral}
\end{eqnarray}
where we now use $u = \pmom/m_{\rm e}$ and $n = n_{\rm e}$:
\begin{eqnarray}
\pedeg = \onethird \int_0^{\pmomz} \frac{\pmom^2}{m_{\rm e}} n_{\rm e}(\pmom) d\pmom.
\end{eqnarray}
Substituting $n_{\rm e}$ from Eq.(\ref{equ:ne}) and integrating gives:
\begin{eqnarray}
\pedeg &=& \onethird \int_0^{\pmomz} \frac{\pmom^2}{m_{\rm e}} \frac{2}{h^3} 4\pi \pmom^2 d\pmom.\\
       &=& \frac{8\pi}{15 m_{\rm e} h^3} \pmomz^5.\label{equ:pp1}
\end{eqnarray}
Now we make use of Eq.(\ref{equ:momz}) and recast the electron number
density in terms of the gas density (lectures)
\begin{equation}
n_{\rm e} = \frac{\rho}{\mue m_{\rm H}}.\label{equ:ne2}
\end{equation}
Substituting these in Eq.~(\ref{equ:pp1}) gives
\begin{eqnarray}
\pedeg &=& \frac{8\pi}{15 m_{\rm e} h^3} \left( \frac{3n_{\rm e}h^3}{8\pi} \right)^{5/3},\\
       &=& \frac{8\pi}{15 m_{\rm e} h^3} \left( \frac{3 h^3}{8\pi} \frac{\rho}{\mue \mH} \right)^{5/3}.
\end{eqnarray}
Rearranging this yields the desired formula:
\begin{equation}
\pedeg = \frac{h^2}{20 m_{\rm e}} \left(\frac{3}{\pi} \right)^{2/3} \frac{1}{m_{\rm H}^{5/3}} \left( \frac{\rho}{\mu_{\rm e}} \right)^{5/3} \equiv K_1 \rho^{5/3}.
\end{equation}

In the (ultra-)relativistic case the the velocity of the electrons
approaches the speed of light. Then we replace $u$ in the pressure
integral (\ref{equ:pintegral}) by $c$
\begin{eqnarray}
\perdeg = \onethird \int_0^{\pmomz} c \pmom n_{\rm e}(\pmom) d\pmom.
\end{eqnarray}
Substituting (\ref{equ:ne}) and integrating yields:
\begin{eqnarray}
\perdeg &=& \onethird \int_0^{\pmomz} c \pmom \frac{2}{h^3} 4\pi \pmom^2 d\pmom.\\
       &=& \frac{2\pi c}{3 h^3} \pmomz^4.\label{equ:pp2}
\end{eqnarray}
We again make use of Eqs.(\ref{equ:momz}) and (\ref{equ:ne2}):
\begin{eqnarray}
\perdeg &=& \frac{2\pi c}{3 h^3} \left( \frac{3n_{\rm e}h^3}{8\pi} \right)^{4/3},\\
        &=& \frac{2\pi c}{3 h^3} \left( \frac{3 h^3}{8\pi} \frac{\rho}{\mue \mH} \right)^{4/3}.
\end{eqnarray}
Rearranging this yields the desired formula:
\begin{equation}
\perdeg = \frac{hc}{8} \left(\frac{3}{\pi} \right)^{1/3} \frac{1}{m_{\rm H}^{4/3}} \left( \frac{\rho}{\mu_{\rm e}} \right)^{4/3} \equiv K_2 \rho^{4/3}.\label{equ:perdeg}
\end{equation}

To compare the gas pressure and degenerate electron pressure in the
Sun, we compute the ratio $\pedeg/\pgas$, where
\begin{equation}
\pgas = \frac{\cal R}{\mu} \rho T,
\end{equation}
and where the specific gas constant is given by ${\cal R} = k/m_{\rm
  H}$,
\begin{equation}
\frac{\pedeg}{\pgas} = \frac{h^2}{20 m_{\rm e}} \left(\frac{3}{\pi} \right)^{2/3} \frac{1}{m_{\rm H}^{5/3}} \frac{\mu}{\mue^{5/3}} \frac{\rho^{2/3}}{{\cal R} T}.
\end{equation}
Inserting the numerical values yields
\begin{equation}
\frac{\pedeg}{\pgas} \approx 0.11,
\end{equation}
which means that electron degeneracy is beginning to have an effect in
the solar centre, although gas pressure still dominates.


{\bf Problem 2:} Derive the mass-radius relation of white dwarfs
assuming non-relativistic degenerate electron gas using the Lane-Emden
equation. Derive the Chandrasekhar mass using the Lane-Emden
equation. Hint: recall the for the equation of state of
non-relativistic (relativistic) degenerate electron gas the polytropic
index is $n={3 \over 2}$ ($n=3$), and assume that hydrogen has been
depleted so that $\mu_{\rm e} = 2$.

{\bf Solution:} Let us first recall the Lane-Emden equation:
\begin{equation}
\frac{1}{\xi^2} \frac{d}{d\xi} \left(\xi^2 \frac{d\theta}{d\xi} \right) = -\theta^n.\label{equ:LaneEmden}
\end{equation}
where $\xi = \alpha r$ and $\rho = \rhoc \theta^n$ where $n$ is the
polytropic index related to the adiabatic index via
\begin{equation}
\gamma = 1 + \frac{1}{n}.
\end{equation}
The mass of the star is given by:
\begin{equation}
M = \int_0^R 4\pi r^2 \rho dr.
\end{equation}
We now make a substitution of variables $r = \xi/\alpha$, $dr =
d\xi/\alpha$, and $\rho = \rhoc \theta^n$, and set the integration
limits to $(0,\xis)$ where $\xis$ is the surface corresponding to the
first zero-crossing of the solution of Eq.~(\ref{equ:LaneEmden}). We
obtain:
\begin{equation}
M = \frac{4\pi \rhoc}{\alpha^3} \int_0^{\xis} \xi^2 \theta^n d\xi.
\end{equation}
Now we substitute the Lane-Emden equation, Eq.~(\ref{equ:LaneEmden}),
to find:
\begin{equation}
M = -\frac{4\pi \rhoc}{\alpha^3} \int_0^{\xis} \frac{d}{d\xi} \left(\xi^2 \frac{d\theta}{d\xi} \right) d\xi.\label{equ:M1}
\end{equation}
We may also write $\xis = \alpha R$, where $R$ is the radius of the
star to eliminate $\alpha$. Furthermore, it is easy to integrate
Eq.~(\ref{equ:M1}) bearing in mind that $d\theta/d\xi =0$ at $\xi=0$:
\begin{equation}
M = -4\pi \rhoc \left(\frac{R}{\xis}\right)^3 \int_0^{\xis} \frac{d}{d\xi} \left(\xi^2 \frac{d\theta}{d\xi} \right) d\xi = -4\pi \rhoc \left(\frac{R}{\xis}\right)^3 \left(\xis^2 \left.\frac{d\theta}{d\xi}\right|_{\xis} \right) = 4\pi \rhoc R^3 \left(- \frac{1}{\xis} \left.\frac{d\theta}{d\xi}\right|_{\xis} \right).\label{equ:M2}
\end{equation}
Now we recall from the homework that
\begin{equation}
\alpha^2 = \frac{4\pi G}{K(n+1)} \rhoc^{1-\frac{1}{n}}, \ \ \mbox{and}\ \ p = K\rho^{1-\frac{1}{n}}.
\end{equation}
We eliminate $\alpha$ with the definition of $\xis$ and $\rhoc$ using
Eq.~(\ref{equ:M2}):
\begin{equation}
\frac{\xis^2}{R^2} = \frac{4\pi G}{K(n+1)} \left[ \frac{M}{4\pi R^3 \left(-\frac{1}{\xis} \left.\frac{d\theta}{d\xi}\right|_{\xis} \right)}  \right]^{1-\frac{1}{n}}.\label{equ:xi2R2}
\end{equation}
Assuming that we have solved the Lane-Emden equation we know $\xis$
and $-\frac{1}{\xis} \left.\frac{d\theta}{d\xi}\right|_{\xis}$ and
identifying that everything except $M$ and $R$ are constants, we find:
\begin{eqnarray}
R^{-2} &\propto& \frac{M^{1-\frac{1}{n}}}{R^{3-\frac{3}{n}}}, \nonumber \\
R      &\propto& M^{\frac{n-1}{n-3}}.
\end{eqnarray}
Inserting $n=\threehalfs$ for non-relativistic electron gas gives
\begin{eqnarray}
R \propto M^{-1/3},
\end{eqnarray}
which shows that the radius of the white dwarf decreases with mass.
If we use $n=3$ corresponding to relativistic electron gas, we see
that the radius approaches zero suggesting that something interesting
happens at this limit.

Going back to Eq.~(\ref{equ:xi2R2}) and substituting $n=3$, we find
that $R$ drops out from the equation. Furthermore, solving for $M$
yields
\begin{equation}
M = 4\pi \left(\frac{K}{\pi G}\right)^{3/2} \left( - \xis^2 \left.\frac{d\theta}{d\xi}\right|_{\xis} \right).
\end{equation}
Identifying that $K=K_2$ from Eq.~(\ref{equ:perdeg}) and that for a
polytrope of $n=3$, $-\xis^2 \left.\frac{d\theta}{d\xi}\right|_{\xis}
= 2.02$, we get
\begin{equation}
M \approx 2.89\cdot 10^{30}~\mbox{kg} = 1.45M_\odot \equiv M_{\rm Ch}.
\end{equation}


{\bf Problem 3:} Convince yourself that normal stars have a built-in
\emph{thermostat} that allows them to maintain thermal stability over
very long periods of time. Show also that the opposite is typically
true for degenerate electron gas.

{\bf Solution:} Assume a star in hydrostatic equilibrium. Then the
central pressure from a polytropic model is given by
\begin{equation}
\ppc = (4\pi)^{1/3} B_n G M^{2/3} \rhoc^{4/3},
\end{equation}
where $B_n$ is a constant depending on the polytropic index $n$. The
relation between $p_{\rm c}$ and $\rho_{\rm c}$ can be recast as
\begin{equation}
\frac{d\ppc}{\ppc} = \frac{4}{3}\frac{d\rhoc}{\rhoc}.
\end{equation}
The pressure, density, and temperature are linked through
the equation of state
\begin{equation}
\frac{d\ppc}{\ppc} = a\frac{d\rhoc}{\rhoc} + b\frac{d\TTc}{\TTc}.
\end{equation}
Combining these equations gives:
\begin{equation}
\left(\frac{4}{3} - a \right) \frac{d\rhoc}{\rhoc} = b\frac{d\TTc}{\TTc}.
\end{equation}
From here we see that for $a<\fourthirds$, the signs of density and
temperature changes are the same. That it, a compression (expansion)
leads to an increase (decrease) of temperature. This is the case for
ideal gas where $a=b=1$.

In this case a compression would lead to an increase in $\rhoc$ and
$\TTc$ and hence increased nuclear energy production and further
increase in $\TTc$. However, then the star expands lowering $\rhoc$
and $\TTc$ such that a new equilibrium is reached. This is the
thermostat mechanims working in non-degenerate stars.

In degenerate stars $a\gtrsim \fourthirds$ and $0<b\ll1$. If the star
expands because its internal energy increases, e.g., by nuclear
reactions the density decreases but the temperature
\emph{increases}. This leads to stronger nuclear energy release, even
higher temperature, and a runaway thermonuclear instability.

This is observed in some close binary systems where a main-sequence or
giant star orbits a white dwarf. The former fills its Roche lobe and
mass (mostly hydrogen) accretes on the white dwarf and forms an
atmosphere. The atmosphere heats up and at some point a critical point
is exceeded and runaway fusion happens. This can either expel the
atmosphere or destroy the white dwarf. Repeated outbursts are
sometimes observed from the former type with periods of decades. These
stars are referred to as recurrent novae. An example is the star T
Coronae Borealis that should have an outburst this year! The latter
case where the white dwarf is destroyed are classified as type Ia
supernovae.


%\newpage
{\bf Useful physical constants}
\begin{itemize}
  \item $R_{\odot} = 696 \times 10^6\,{\rm m}$
  \item $M_{\odot} = 1.989 \times 10^{30}\,{\rm kg}$
  \item $L_{\odot} = 3.83 \times 10^{26}$~W
  \item $T^{\rm eff}_{\odot} = 5777\,{\rm K}$
  \item $1\,{\rm AU} = 1.496 \times 10^8\,{\rm km}$
  \item $c = 2.997 \times 10^8\,{\rm m/s}$
  \item $G = 6.674 \times 10^{-11}$~Nm$^2$/kg$^2$
  \item $k = 1.38\cdot10^{-23}$~J/K
  \item $m_{\rm e} = 9.11\cdot10^{-31}$~kg
  \item $m_{\rm H} = 1.67\cdot10^{-27}$~kg
\end{itemize}









\end{document}
