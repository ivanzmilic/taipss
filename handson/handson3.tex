\documentclass[12pt]{article}
\special{papersize=210mm,297mm}
\usepackage[top=1.5cm,bottom=1.5cm,left=2cm,right=2cm]{geometry}
\usepackage[skip=12pt plus1pt, indent=0pt]{parskip}

\usepackage{fancyhdr}
\usepackage{graphicx}
\usepackage{amsmath}
\usepackage{cancel}
\usepackage{amssymb}
\usepackage{eurosym}
\usepackage{multicol}
\usepackage{rotate}
\usepackage{tabularx}
\usepackage{floatrow}
\usepackage{color}
\usepackage{colortbl}
\usepackage{braket}
\usepackage[shortcuts]{extdash}
\setlength{\headheight}{15.2pt}
\pagestyle{fancy}
\renewcommand{\headrulewidth}{0.5pt}
\renewcommand{\footrulewidth}{0.5pt}

\fancyhf{}
%%%% colors
\definecolor{orange}{RGB}{250,167,12}
\definecolor{yellow}{RGB}{246,250,12}
\definecolor{green}{RGB}{128,238,1}
\definecolor{green2}{RGB}{0,250,154}
\definecolor{black}{RGB}{0,0,0}
\definecolor{blue}{RGB}{0,0,255}
\definecolor{red}{RGB}{255,0,0}
\definecolor{white}{RGB}{255,255,255}
\definecolor{burlywood1}{RGB}{255,211,155}
\definecolor{chocolate1}{RGB}{255,127,36}
\definecolor{sepia}{RGB}{94,38,18}
\newcommand{\blue}[1]{\textcolor{blue}{#1}}
\newcommand{\sepia}[1]{\textcolor{sepia}{#1}}
\newcommand{\red}[1]{\textcolor{red}{#1}}
\newcommand{\green}[1]{\textcolor{green}{#1}}
\newcommand{\yellow}[1]{\textcolor{yellow}{#1}}
\newcommand{\orange}[1]{\textcolor{orange}{#1}}
%%%% mathematical definitions
\DeclareMathOperator{\Tr}{Tr}
\def\npab{\noindent \textbullet ~}
\def\npa{\noindent}
\def\npat{\noindent \textcolor{blue}{$\blacktriangleright$} ~}
\def\npac{\noindent $\circledast$ ~}
\def\nn{{\bf \nabla}}
\def\df{{\rm d}}
\def\cro{\times}
\def\ip{i^{'}}
\def\jp{j^{'}}
\def\kp{k^{'}}
\def\deg{$^{\circ}$}
\def\rhoc{\textcolor{red}{\rho}}
\def\uic{\textcolor{red}{u_i}}
\def\ujc{\textcolor{red}{u_j}}
\def\pgc{\textcolor{red}{P_{\rm g}}}
\newcommand{\ve}[1]{{\rm\bf {#1}}}
\def\ez{\ve{e}_z}
\def\ey{\ve{e}_y}
\def\ex{\ve{e}_x}
\def\ezero{\ve{e}_{0}}
\def\eplu{\ve{e}_{+1}}
\def\emin{\ve{e}_{-1}}
\def\definition{:=^{\!\!\!\!\!\!\!\textrm{def}}}
\newcommand{\bg}[1]{{\boldsymbol {#1}}}
\newcommand*\tavg[1]{%
  \hbox{%
    \vbox{%
      \hrule height 0.5pt % The actual bar
      \kern0.5ex%         % Distance between bar and symbol
      \hbox{%
        \kern-0.1em%      % Shortening on the left side
        \ensuremath{#1}%
        \kern-0.1em%      % Shortening on the right side
      }%
    }%
  }%
}
\title{Hands-on exercises 3: Equations of stellar structure and
  polytropic equation of state}
\author{P. K\"{a}pyl\"{a}, I. Mili\'{c}}
\date{\today}
%%%%
\begin{document}
\maketitle


These are mostly analytical exercises.

{\bf Problem 1:} Let us consider the set of static stellar structure
equations in equilibrium with fixed chemical composition:
\begin{eqnarray}
  \frac{dP}{dr} &=& -\rho \frac{Gm}{r^2},\\
  \frac{dm}{dr} &=& 4\pi r^2\rho, \\
  \frac{dT}{dr} &=& -\frac{3}{4ac}\frac{\kappa\rho}{T^3}\frac{F}{4\pi r^2},\\
  \frac{dF}{dr} &=& 4\pi r^2 \rho q.
\end{eqnarray}
These need to be supplemented by an equation of state:
\begin{eqnarray}
P = \frac{\cal R}{\mu_{\rm I}}\rho T + P_e + \textstyle{1\over3} a T^4,
\end{eqnarray}
and the equations for the opacity and energy production rate:
\begin{eqnarray}
  \kappa &=& \kappa_0 \rho^a T^b,\\
  q &=& q_0 \rho^m T^n.
\end{eqnarray}
Despite the explicit and implicit simplifications this is still a
quite complex set of equations. Discuss the meaning of each of the
equations and the issues in solving them in more detail. What can you
say about the boundary conditions we need to apply?


{\bf Problem 2:} Use the equations of hydrostatic equilibrium and mass
conservation:
\begin{eqnarray}
  \frac{dP}{dr} = -\rho \frac{Gm}{r^2},\ \ \frac{dm}{dr} = 4\pi r^2\rho, 
\end{eqnarray}
assuming a polytropic equation of state
\begin{equation}
P = K \rho^\gamma,
\end{equation}
where $\gamma = 1 + \frac{1}{n}$ is the polytropic exponent and $n$ is
the polytropic index, to derive the Lane-Emden equation:
\begin{equation}
\frac{1}{\xi^2} \frac{d}{d\xi} \left(\xi^2 \frac{d\theta}{d\xi} \right) = -\theta^n.\label{equ:LaneEmden}
\end{equation}
Here $\theta$ and $\xi$ are the non-dimensional density and radius
defined as
\begin{equation}
\rho = \rho_{\rm c} \theta^n, \ \ \mbox{and}\ \ r = \alpha\xi,
\end{equation}
and where $\alpha^2$ equals a constant that arises in the derivation
of Eq.(\ref{equ:LaneEmden}).

The boundary conditions at $\xi=0$ for the Lane-Emden equation are:
\begin{equation}
\theta = 1,\ \mbox{and}\ \frac{d\theta}{d\xi} = 0.
\end{equation}
What do these boundary conditions correspond to?

{\bf Useful physical constants}
\begin{itemize}
  \item $R_{\odot} = 696 \times 10^6\,{\rm m}$
  \item $M_{\odot} = 1.989 \times 10^{30}\,{\rm kg}$
  \item $L_{\odot} = 3.83 \times 10^{26}$~W
  \item $T^{\rm eff}_{\odot} = 5777\,{\rm K}$
  \item $1\,{\rm AU} = 1.496 \times 10^8\,{\rm km}$
  \item $c = 2.997 \times 10^8\,{\rm m/s}$
  \item $G = 6.674 \times 10^{-11}$~Nm$^2$/kg$^2$
  \item $k = 1.38\cdot10^{-23}$~J/K
  \item $m_{\rm H} = 1.67\cdot10^{-27}$~kg
\end{itemize}









\end{document}
