\documentclass[12pt]{article}
\special{papersize=210mm,297mm}
\usepackage[top=1.5cm,bottom=1.5cm,left=2cm,right=2cm]{geometry}
\usepackage[skip=12pt plus1pt, indent=0pt]{parskip}

\usepackage{fancyhdr}
\usepackage{graphicx}
\usepackage{amsmath}
\usepackage{cancel}
\usepackage{amssymb}
\usepackage{eurosym}
\usepackage{multicol}
\usepackage{rotate}
\usepackage{tabularx}
\usepackage{floatrow}
\usepackage{color}
\usepackage{colortbl}
\usepackage{braket}
\usepackage{bm}
\usepackage[shortcuts]{extdash}
\setlength{\headheight}{15.2pt}
\pagestyle{fancy}
\renewcommand{\headrulewidth}{0.5pt}
\renewcommand{\footrulewidth}{0.5pt}

\fancyhf{}
%%%% colors
\definecolor{orange}{RGB}{250,167,12}
\definecolor{yellow}{RGB}{246,250,12}
\definecolor{green}{RGB}{128,238,1}
\definecolor{green2}{RGB}{0,250,154}
\definecolor{black}{RGB}{0,0,0}
\definecolor{blue}{RGB}{0,0,255}
\definecolor{red}{RGB}{255,0,0}
\definecolor{white}{RGB}{255,255,255}
\definecolor{burlywood1}{RGB}{255,211,155}
\definecolor{chocolate1}{RGB}{255,127,36}
\definecolor{sepia}{RGB}{94,38,18}
\newcommand{\blue}[1]{\textcolor{blue}{#1}}
\newcommand{\sepia}[1]{\textcolor{sepia}{#1}}
\newcommand{\red}[1]{\textcolor{red}{#1}}
\newcommand{\green}[1]{\textcolor{green}{#1}}
\newcommand{\yellow}[1]{\textcolor{yellow}{#1}}
\newcommand{\orange}[1]{\textcolor{orange}{#1}}
%%%% mathematical definitions
\newcommand{\kkk}{\bm{k}}
\newcommand{\rrr}{\bm{r}}
\newcommand{\uuu}{\bm{u}}
\newcommand{\FFF}{\bm{F}}
\newcommand{\gggg}{\bm{g}}
\DeclareMathOperator{\Tr}{Tr}
\def\pd{\partial}
\def\npab{\noindent \textbullet ~}
\def\npa{\noindent}
\def\npat{\noindent \textcolor{blue}{$\blacktriangleright$} ~}
\def\npac{\noindent $\circledast$ ~}
\def\nn{{\bf \nabla}}
\def\df{{\rm d}}
\def\cro{\times}
\def\ip{i^{'}}
\def\jp{j^{'}}
\def\kp{k^{'}}
\def\deg{$^{\circ}$}
\def\rhoc{\textcolor{red}{\rho}}
\def\uic{\textcolor{red}{u_i}}
\def\ujc{\textcolor{red}{u_j}}
\def\pgc{\textcolor{red}{P_{\rm g}}}
\newcommand{\ve}[1]{{\rm\bf {#1}}}
\def\ez{\ve{e}_z}
\def\ey{\ve{e}_y}
\def\ex{\ve{e}_x}
\def\ezero{\ve{e}_{0}}
\def\eplu{\ve{e}_{+1}}
\def\emin{\ve{e}_{-1}}
\def\definition{:=^{\!\!\!\!\!\!\!\textrm{def}}}
\newcommand{\bg}[1]{{\boldsymbol {#1}}}
\newcommand*\tavg[1]{%
  \hbox{%
    \vbox{%
      \hrule height 0.5pt % The actual bar
      \kern0.5ex%         % Distance between bar and symbol
      \hbox{%
        \kern-0.1em%      % Shortening on the left side
        \ensuremath{#1}%
        \kern-0.1em%      % Shortening on the right side
      }%
    }%
  }%
}
\title{Hands-on exercises 10 solution: Gravity waves}
\author{P. K\"{a}pyl\"{a}, I. Mili\'{c}}
\date{\today}
%%%%
\begin{document}
\maketitle

{\bf Problem 1:} Add gravity to the system that was used to
demonstrate sound waves on the lectures and derive the equations
governing internal gravity waves or g modes.

{\bf Solution:} We first recall that we still have the pressure
gradient which led to the sound waves. We will also assume that the
gradients of the 0-quantities are negligible compared to the gradients
of the fluctuations. The fluctuating contribution to the gravitational
potential is also omitted.

Then the linearized equations read:
\begin{equation}
\rho' + \bm\nabla\bm\cdot (\rho_0\delta\rrr') = 0,\label{equ:cont}
\end{equation}
\begin{equation}
\rho_0 \frac{\pd \uuu'}{\pd t} = -\bm\nabla p' + \rho_0'\gggg.\label{equ:mom}% + \rho'\gggg'.
\end{equation}
We assume from the start that the solution is a plane wave:
\begin{equation}
\exp [i(\kkk\bm\cdot\rrr - \omega t)].
\end{equation}
Let the $r$ coordinate increase outwards so that $\gggg_0 = -g_0
\hat{\bm e}_r$, and
\begin{equation}
\bm\nabla p_0 = \frac{{\rm d}p_0}{{\rm d}r} \hat{\bm e}_r,\ \ \mbox{and}\ \ \bm\nabla \rho_0 = \frac{{\rm d}\rho_0}{{\rm d}r} \hat{\bm e}_r.
\end{equation}
Let us further separate the directions along (radial) and
perpendicular (horizontal) to the gravity:
\begin{eqnarray}
\delta\rrr &=& \xi_r \hat{\bm e}_r + \bm\xi_{\rm h},\\
\kkk &=& k_r \hat{\bm e}_r + \kkk_{\rm h}.
\end{eqnarray}
Inserting these into Eq.~(\ref{equ:mom}) give:
\begin{eqnarray}
-\rho_0 \omega^2 \xi_r &=& -ik_rp' - \rho' g_0,\label{equ:rad}\\
-\rho_0 \omega^2 \bm\xi_{\rm h} &=& -i \kkk_{\rm h} p'.
\end{eqnarray}
The continuity equation (\ref{equ:cont}) gives:
\begin{equation}
\rho' + \rho_0 i k_r \xi_r + \rho_0 i \kkk_{\rm h}\bm\cdot\bm\xi_{\rm h} = 0.
\end{equation}
The two previous equations combine to:
\begin{equation}
p' = \frac{\omega^2}{k_{\rm h}^2} (\rho' + \rho_0 i k_r \xi_r).\label{equ:pprime}
\end{equation}
Using this in Eq.~(\ref{equ:rad}) yields:
\begin{equation}
-\rho_0 \omega^2 \xi_r = -i \frac{k_r}{k_{\rm h}^2}\omega^2 \rho' + \omega^2 \rho_0 \frac{k_r^2}{k_{\rm h}^2}\xi_r -\rho' g_0.
\end{equation}
For very small frequencies $\omega$ the first term proportional to
$\rho'$ is small compared to the buoyancy term and we therefore omit
it. Then the equation reduces to:
\begin{equation}
\rho_0 \omega^2 \left(1 - \frac{k_r^2}{k_{\rm h}^2} \right)\xi_r = \rho' g_0.\label{equ:rhoom}
\end{equation}
The term on the right hand side is the driving of density
perturbations by the buoyancy force, whereas the left hand side gives
the vertical acceleration. Note the term with the wavenumbers: this
arises from the pressure force because a radial displacement means
that matter has to be pushed aside horizontally which adds up to the
total inertia. This effect is particularly large for large horizontal
length scales with $k_{\rm h} \ll k_r$.

To arrive at the dispersion relation we again make use of the
adiabatic relation:
\begin{equation}
\delta p = \gamma_{1,0} \frac{p_0}{\rho_0} \delta \rho.
\end{equation}
Note that the fluctuations are Lagrangian, i.e., they follow the fluid
elements and in Eulerian form where we sit in a fixed position they
read $\delta p = p' + \delta\rrr\bm\cdot\bm\nabla p_0$ and $\delta
\rho = \rho' + \delta\rrr\bm\cdot\bm\nabla\rho_0$. Therefore,
\begin{equation}
\rho' = c_0^{-2} p' + \rho_0 \delta\rrr\bm\cdot \left( \frac{1}{\gamma_{1,0} p_0}\bm\nabla p_0 - \frac{1}{\rho_0}\bm\nabla \rho_0 \right).\label{equ:rhoprime}
\end{equation}
Using Eq.~(\ref{equ:pprime}) we find that
\begin{equation}
\frac{c_0^{-2}p'}{\rho'} \approx \frac{\omega^2}{c_0^2k_{\rm h}},
\end{equation}
where $c_0^2k_{\rm h}$ is the frequency corresponding to sound
waves. The frequencies $\omega$ that we are interested in are much
lower so we can therefore omit the term proportional to $p'$.
Substitution of Eq.~(\ref{equ:rhoprime}) without this term to
Eq.~(\ref{equ:rhoom}) yields:
\begin{equation}
\omega^2 \left( 1 + \frac{k_r^2}{k_{\rm h}^2} \right) \xi_r = N^2 \xi_r,
\end{equation}
where
\begin{equation}
N^2 = g_0 \left( \frac{1}{\gamma_{1,0}} \frac{{\rm d}\ln p_0}{{\rm d}r} - \frac{{\rm d}\ln \rho_0}{{\rm d}r} \right),
\end{equation}
is the squared \emph{Brunt-V\"ais\"al\"a} frequency. The
Brunt-V\"ais\"al\"a frequency related to the convective stability of
the fluid: for $N^2>0$ the stratification is stable and for $N^2<0$ it
is unstable. This is more clearly seen from the dispersion relation of
gravity waves:
\begin{equation}
\omega^2 = \frac{N^2}{1 + k_r^2/k_{\rm h}^2}.
\end{equation}
The frequency of the gravity waves coincides with $N^2$ ($>0$) when
$k_{\rm h} \rightarrow \infty$. For larger $k_{\rm h}$ the frequency is
decreased. If $N^2<0$, the solution grows exponentially in time
corresponding to the convective instability.


%% {\bf Problem 2:} Consider an infinitely deep fluid with density
%% $\rho_0$ with a free surface. Assume incopressibility and that the
%% pressure at the surface is constant. Derive the equations governing
%% surface gravity waves or the f mode.


%% {\bf Solution:} Incompressibility means that the density is constant,
%% i.e., $\rho = \rho_0$. Then the continuity equation reads:
%% \begin{equation}
%%   \bm\nabla\bm\cdot\uuu = 0.
%% \end{equation}
%% Assume a constant downward-pointing gravity and note that because
%% $\rho'=0$, also $\Phi'=0$. The equation of motion reduces to
%% \begin{equation}
%% \rho_0 \frac{\pd \uuu}{\pd t} = - \bm\nabla p',
%% \end{equation}
%% which reduces further to
%% \begin{equation}
%% \nabla^2 p' =0,\label{equ:del2p}
%% \end{equation}
%% after taking the divergence. Introduce horizontal coordinate $x$ and
%% vertical coordinate $z$. The latter increases downward and the surface
%% is at $z=0$. We seek a solution for a wave propagating along the
%% surface in the horizontal direction. It has the general form:
%% \begin{equation}
%% p' = f(z) \cos(k_{\rm h}x - \omega t),\label{equ:pprime2}
%% \end{equation}
%% where $f$ needs to be determined. Substituting (\ref{equ:pprime2})
%% into (\ref{equ:del2p}) gives:
%% \begin{equation}
%% \frac{\pd^2 f}{\pd z^2} = k_{\rm h}^2 f,
%% \end{equation}
%% or
%% \begin{equation}
%% f = a \exp(-k_{\rm h}z) + b \exp(k_{\rm h}z),
%% \end{equation}
%% where $b$ has to be zero (\blue{Why?}).

%% The boundary condition at the surface is that the pressure is
%% constant. This means that the Lagrangian pressure perturbation
%% vanishes, or
%% \begin{equation}
%% 0 = \delta p = p' + \delta\rrr\bm\cdot\bm\nabla p_0 = p' + \xi_z \rho_0 g_0,
%% \end{equation}
%% where $\xi_z$ is the vertical component of the displacement.

%\newpage
{\bf Useful physical constants}
\begin{itemize}
  \item $R_{\odot} = 696 \times 10^6\,{\rm m}$
  \item $M_{\odot} = 1.989 \times 10^{30}\,{\rm kg}$
  \item $L_{\odot} = 3.83 \times 10^{26}$~W
  \item $T^{\rm eff}_{\odot} = 5777\,{\rm K}$
  \item $1\,{\rm AU} = 1.496 \times 10^8\,{\rm km}$
  \item $c = 2.997 \times 10^8\,{\rm m/s}$
  \item $G = 6.674 \times 10^{-11}$~Nm$^2$/kg$^2$
  \item $k = 1.38\cdot10^{-23}$~J/K
  \item $m_{\rm e} = 9.11\cdot10^{-31}$~kg
  \item $m_{\rm H} = 1.67\cdot10^{-27}$~kg
\end{itemize}

\end{document}
