\documentclass[12pt]{article}
\special{papersize=210mm,297mm}
\usepackage[top=1.5cm,bottom=1.5cm,left=2cm,right=2cm]{geometry}
\usepackage[skip=12pt plus1pt, indent=0pt]{parskip}

\usepackage{fancyhdr}
\usepackage{graphicx}
\usepackage{amsmath}
\usepackage{cancel}
\usepackage{amssymb}
\usepackage{eurosym}
\usepackage{multicol}
\usepackage{rotate}
\usepackage{tabularx}
\usepackage{floatrow}
\usepackage{color}
\usepackage{colortbl}
\usepackage{braket}
\usepackage{bm}
\usepackage[shortcuts]{extdash}
\setlength{\headheight}{15.2pt}
\pagestyle{fancy}
\renewcommand{\headrulewidth}{0.5pt}
\renewcommand{\footrulewidth}{0.5pt}

\fancyhf{}
%%%% colors
\definecolor{orange}{RGB}{250,167,12}
\definecolor{yellow}{RGB}{246,250,12}
\definecolor{green}{RGB}{128,238,1}
\definecolor{green2}{RGB}{0,250,154}
\definecolor{black}{RGB}{0,0,0}
\definecolor{blue}{RGB}{0,0,255}
\definecolor{red}{RGB}{255,0,0}
\definecolor{white}{RGB}{255,255,255}
\definecolor{burlywood1}{RGB}{255,211,155}
\definecolor{chocolate1}{RGB}{255,127,36}
\definecolor{sepia}{RGB}{94,38,18}
\newcommand{\blue}[1]{\textcolor{blue}{#1}}
\newcommand{\sepia}[1]{\textcolor{sepia}{#1}}
\newcommand{\red}[1]{\textcolor{red}{#1}}
\newcommand{\green}[1]{\textcolor{green}{#1}}
\newcommand{\yellow}[1]{\textcolor{yellow}{#1}}
\newcommand{\orange}[1]{\textcolor{orange}{#1}}
%%%% mathematical definitions
\DeclareMathOperator{\Tr}{Tr}
\def\pd{\partial}
\def\npab{\noindent \textbullet ~}
\def\npa{\noindent}
\def\npat{\noindent \textcolor{blue}{$\blacktriangleright$} ~}
\def\npac{\noindent $\circledast$ ~}
\def\nn{{\bf \nabla}}
\def\df{{\rm d}}
\def\cro{\times}
\def\ip{i^{'}}
\def\jp{j^{'}}
\def\kp{k^{'}}
\def\deg{$^{\circ}$}
\def\rhoc{\textcolor{red}{\rho}}
\def\uic{\textcolor{red}{u_i}}
\def\ujc{\textcolor{red}{u_j}}
\def\pgc{\textcolor{red}{P_{\rm g}}}
\newcommand{\ve}[1]{{\rm\bf {#1}}}
\def\ez{\ve{e}_z}
\def\ey{\ve{e}_y}
\def\ex{\ve{e}_x}
\def\ezero{\ve{e}_{0}}
\def\eplu{\ve{e}_{+1}}
\def\emin{\ve{e}_{-1}}
\def\definition{:=^{\!\!\!\!\!\!\!\textrm{def}}}
\newcommand{\bg}[1]{{\boldsymbol {#1}}}
\newcommand*\tavg[1]{%
  \hbox{%
    \vbox{%
      \hrule height 0.5pt % The actual bar
      \kern0.5ex%         % Distance between bar and symbol
      \hbox{%
        \kern-0.1em%      % Shortening on the left side
        \ensuremath{#1}%
        \kern-0.1em%      % Shortening on the right side
      }%
    }%
  }%
}
\title{Hands-on exercises 7: Degenerate electron gas, properties of white dwarfs, and thermonuclear instability}
\author{P. K\"{a}pyl\"{a}, I. Mili\'{c}}
\date{\today}
%%%%
\begin{document}
\maketitle

{\bf Problem 1:} Use the Heisenberg uncertainity principle and Pauli
exclusion principle to derive the equations of state for
non-relativistic and relativistic degenerate electron gas:
\begin{equation}
p_{\rm e,deg} = \frac{h^2}{20 m_{\rm e}} \left(\frac{3}{\pi} \right)^{2/3} \frac{1}{m_{\rm H}^{5/3}} \left( \frac{\rho}{\mu_{\rm e}} \right)^{5/3},
\end{equation}
and 
\begin{equation}
p_{\rm e,r-deg} = \frac{hc}{8} \left(\frac{3}{\pi} \right)^{1/3} \frac{1}{m_{\rm H}^{4/3}} \left( \frac{\rho}{\mu_{\rm e}} \right)^{4/3},
\end{equation}
where $h = 6.626\cdot 10^{-34}$J~Hz$^{-1}$ is the Planck constant,
$m_{\rm e}$ is the mass of the electron, $m_{\rm H}$ is the atomic
mass unit, and $\mu_{\rm e}^{-1}$ is the average number of free
electron per nucleon.

Compare the degenerate electron pressure and the ``normal'' gas
pressure at the centre of the Sun with solar composition of $\mu=0.62$
and $\mu_{\rm e} = 1.17$. Use $\rho_{\rm c}$ and $T_{\rm c}$ for the
central density and temperature of the Sun.

{\bf Problem 2:} Derive the mass-radius relation of white dwarfs
assuming non-relativistic degenerate electron gas using the Lane-Emden
equation. Derive the Chandrasekhar mass using the Lane-Emden
equation. Hint: recall the for the equation of state of
non-relativistic (relativistic) degenerate electron gas the polytropic
index is $n={3 \over 2}$ ($n=3$), and assume that hydrogen has been
depleted so that $\mu_{\rm e} = 2$.

{\bf Problem 3:} Convince yourself that normal stars have a built-in
\emph{thermostat} that allows them to maintain thermal stability over
very long periods of time. Show also that the opposite is typically
true for degenerate electron gas.


%\newpage
{\bf Useful physical constants}
\begin{itemize}
  \item $R_{\odot} = 696 \times 10^6\,{\rm m}$
  \item $M_{\odot} = 1.989 \times 10^{30}\,{\rm kg}$
  \item $L_{\odot} = 3.83 \times 10^{26}$~W
  \item $T^{\rm eff}_{\odot} = 5777\,{\rm K}$
  \item $1\,{\rm AU} = 1.496 \times 10^8\,{\rm km}$
  \item $c = 2.997 \times 10^8\,{\rm m/s}$
  \item $G = 6.674 \times 10^{-11}$~Nm$^2$/kg$^2$
  \item $k = 1.38\cdot10^{-23}$~J/K
  \item $m_{\rm e} = 9.11\cdot10^{-31}$~kg
  \item $m_{\rm H} = 1.67\cdot10^{-27}$~kg
\end{itemize}









\end{document}
