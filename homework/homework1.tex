\documentclass[12pt]{article}
\special{papersize=210mm,297mm}
\usepackage[top=1.5cm,bottom=1.5cm,left=2cm,right=2cm]{geometry}
\usepackage[skip=12pt plus1pt, indent=0pt]{parskip}

% Usual packages
\usepackage{graphicx}
\usepackage{hyperref}

\usepackage{fancyhdr}
\usepackage{amsmath}
\usepackage{cancel}
\usepackage{amssymb}
\usepackage{eurosym}
\usepackage{multicol}
\usepackage{rotate}
\usepackage{tabularx}
\usepackage{floatrow}
\usepackage{color}
\usepackage{colortbl}
\usepackage{braket}
\usepackage[shortcuts]{extdash}
\setlength{\headheight}{15.2pt}
\pagestyle{fancy}
\renewcommand{\headrulewidth}{0.5pt}
\renewcommand{\footrulewidth}{0.5pt}

\fancyhf{}
%%%% colors
\definecolor{orange}{RGB}{250,167,12}
\definecolor{yellow}{RGB}{246,250,12}
\definecolor{green}{RGB}{128,238,1}
\definecolor{green2}{RGB}{0,250,154}
\definecolor{black}{RGB}{0,0,0}
\definecolor{blue}{RGB}{0,0,255}
\definecolor{red}{RGB}{255,0,0}
\definecolor{white}{RGB}{255,255,255}
\definecolor{burlywood1}{RGB}{255,211,155}
\definecolor{chocolate1}{RGB}{255,127,36}
\definecolor{sepia}{RGB}{94,38,18}
\newcommand{\blue}[1]{\textcolor{blue}{#1}}
\newcommand{\sepia}[1]{\textcolor{sepia}{#1}}
\newcommand{\red}[1]{\textcolor{red}{#1}}
\newcommand{\green}[1]{\textcolor{green}{#1}}
\newcommand{\yellow}[1]{\textcolor{yellow}{#1}}
\newcommand{\orange}[1]{\textcolor{orange}{#1}}
%%%% mathematical definitions
\DeclareMathOperator{\Tr}{Tr}
\def\npab{\noindent \textbullet ~}
\def\npa{\noindent}
\def\npat{\noindent \textcolor{blue}{$\blacktriangleright$} ~}
\def\npac{\noindent $\circledast$ ~}
\def\nn{{\bf \nabla}}
\def\df{{\rm d}}
\def\cro{\times}
\def\ip{i^{'}}
\def\jp{j^{'}}
\def\kp{k^{'}}
\def\deg{$^{\circ}$}
\def\rhoc{\textcolor{red}{\rho}}
\def\uic{\textcolor{red}{u_i}}
\def\ujc{\textcolor{red}{u_j}}
\def\pgc{\textcolor{red}{P_{\rm g}}}
\newcommand{\ve}[1]{{\rm\bf {#1}}}
\def\ez{\ve{e}_z}
\def\ey{\ve{e}_y}
\def\ex{\ve{e}_x}
\def\ezero{\ve{e}_{0}}
\def\eplu{\ve{e}_{+1}}
\def\emin{\ve{e}_{-1}}
\def\definition{:=^{\!\!\!\!\!\!\!\textrm{def}}}
\newcommand{\bg}[1]{{\boldsymbol {#1}}}
\newcommand*\tavg[1]{%
  \hbox{%
    \vbox{%
      \hrule height 0.5pt % The actual bar
      \kern0.5ex%         % Distance between bar and symbol
      \hbox{%
        \kern-0.1em%      % Shortening on the left side
        \ensuremath{#1}%
        \kern-0.1em%      % Shortening on the right side
      }%
    }%
  }%
}
\title{Theoretical Astrophysics: Physics of Sun and Stars\\
Homework 1}
\author{P. K\"{a}pyl\"{a} \& I. Mili\'{c}}
\date{\today}
%%%%
\begin{document}
\maketitle

\textbf{Deadline for this homework is 30/05/2025 23:59}


{\bf Problem 1:} Calculate the mean temperature of the Sun. Compare it
with the effective temperature of the Sun. Hint: Use the virial
theorem and bear in mind the ideal gas equation.

{\bf Problem 2:} For a gas in hydrostatic equilibrium with a constant
mean molecular mass and mean temperature, in a constant gravitational
field, show that the pressure falls off exponentially with height
($z$):
\begin{equation}
p = p_0 e^{-z/H}
\end{equation}
where $H$ is the pressure scale height. Find the expression for the
scale height and calculate its value for the Earth's atmosphere and
the solar atmosphere (for the moment, assume the Sun is composed of
pure neutral hydrogen atoms).

{\bf Problem 3:} Calculate the values of the dynamic, thermal, and
nuclear timescales for the Sun and for a red giant (you can look up
physical properties of your favorite red giant from the
internet). This is an order of magnitude (OOM) estimation - do not get
too hung-up on the constants. Think about the physical meaning of the
timescales.

{\bf Problem 4:} Stefan-Boltzmann law describes the emergent flux from
a blackbody. That is, total energy per unit surface in unit time,
emitted into the $2\pi$ solid angle (full sphere is $4\pi$ but we are
only looking at the outgoing half).

Planck law describes the intensity of radiation in a black body,
intensity is defined as:

\begin{equation}
I_\lambda = \frac{dE}{dt dS d\Omega d\lambda \cos \theta}
\end{equation}
where $\Omega$ is the solid angle $d\Omega = \sin \theta d\theta
d\phi$, and $\theta$ and $\phi$ describe the direction of propagation
of radiation. For Planck's law:
\begin{equation}
I_\lambda = \frac{2hc^2}{\lambda^5} \frac{1}{e^{hc/\lambda k T} - 1}.
\end{equation}
Using Planck's law, derive the Stefan-Boltzmann law:
\begin{equation}
F = \frac{dE}{dt dS} = \sigma T^4
\end{equation}


{\bf Problem 5:} The solar luminosity is $L_\odot = 3.83\cdot
10^{26}$~W. Assume that all of the energy for this luminosity is
provided by the ${\rm pp1}$ chain, and that neutrinos carry off 3\% of
the energy. How many neutrinos are produced per second? What is the
neutrino flux (i.e., the number of neutrinos per second per cm$^2$) at
Earth?

The ${\rm pp1}$ chain consists of the reactions:
\begin{eqnarray}
  p + p &\rightarrow& ^2{\rm D} + e^+ + \nu,\ (1.18~{\rm MeV})\nonumber \\
  ^2{\rm D} + p &\rightarrow& ^3{\rm He} + \gamma,\ \ \ \ \ \ (5.49~{\rm MeV}) \nonumber \\
  ^3{\rm He} + ^3{\rm He} &\rightarrow& ^4{\rm He} + 2p.\ \ \ \ \ (12.86~{\rm MeV})
\end{eqnarray}
where the first two reactions need to happen twice for the last
reaction to occur, and where $\nu$ signifies an emitted
neutrino. Energy released (in MeV = $10^6$~eV) is given in brackets
after each reaction.


{\bf Useful physical constants}
\begin{itemize}
  \item $R_{\odot} = 696 \times 10^6\,{\rm m}$
  \item $M_{\odot} = 1.989 \times 10^{30}\,{\rm kg}$
  \item $L_{\odot} = 3.83 \times 10^{26}$~W
  \item $T^{\rm eff}_{\odot} = 5777\,{\rm K}$
  \item $1\,{\rm AU} = 1.496 \times 10^8\,{\rm km}$
  \item $c = 2.997 \times 10^8\,{\rm m/s}$
  \item $G = 6.674 \times 10^{-11}$~Nm$^2$/kg$^2$
  \item $k = 1.38\cdot10^{-23}$~J/K
  \item $m_{\rm H} = 1.67\cdot10^{-27}$~kg
  \item $h = 6.626 \times 10^{-34}$~J~s.
  \item $k = 1.38 \times 10^{-23}$~J/K.
\end{itemize}
\end{document}
