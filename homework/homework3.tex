\documentclass[12pt]{article}
\special{papersize=210mm,297mm}
\usepackage[top=1.5cm,bottom=1.5cm,left=2cm,right=2cm]{geometry}
\usepackage[skip=12pt plus1pt, indent=0pt]{parskip}

% Usual packages
\usepackage{graphicx}
\usepackage{hyperref}

\usepackage{fancyhdr}
\usepackage{amsmath}
\usepackage{amssymb}
\usepackage{eurosym}
\usepackage{multicol}
\usepackage{rotate}
\usepackage{tabularx}
\usepackage{color}
\usepackage{colortbl}
\setlength{\headheight}{15.2pt}
\pagestyle{fancy}
\renewcommand{\headrulewidth}{0.5pt}
\renewcommand{\footrulewidth}{0.5pt}

\fancyhf{}
%%%% colors
\definecolor{orange}{RGB}{250,167,12}
\definecolor{yellow}{RGB}{246,250,12}
\definecolor{green}{RGB}{128,238,1}
\definecolor{green2}{RGB}{0,250,154}
\definecolor{black}{RGB}{0,0,0}
\definecolor{blue}{RGB}{0,0,255}
\definecolor{red}{RGB}{255,0,0}
\definecolor{white}{RGB}{255,255,255}
\definecolor{burlywood1}{RGB}{255,211,155}
\definecolor{chocolate1}{RGB}{255,127,36}
\definecolor{sepia}{RGB}{94,38,18}
\newcommand{\blue}[1]{\textcolor{blue}{#1}}
\newcommand{\sepia}[1]{\textcolor{sepia}{#1}}
\newcommand{\red}[1]{\textcolor{red}{#1}}
\newcommand{\green}[1]{\textcolor{green}{#1}}
\newcommand{\yellow}[1]{\textcolor{yellow}{#1}}
\newcommand{\orange}[1]{\textcolor{orange}{#1}}
%%%% mathematical definitions
\DeclareMathOperator{\Tr}{Tr}
\def\npab{\noindent \textbullet ~}
\def\npa{\noindent}
\def\npat{\noindent \textcolor{blue}{$\blacktriangleright$} ~}
\def\npac{\noindent $\circledast$ ~}
\def\nn{{\bf \nabla}}
\def\df{{\rm d}}
\def\cro{\times}
\def\ip{i^{'}}
\def\jp{j^{'}}
\def\kp{k^{'}}
\def\deg{$^{\circ}$}
\def\rhoc{\textcolor{red}{\rho}}
\def\uic{\textcolor{red}{u_i}}
\def\ujc{\textcolor{red}{u_j}}
\def\pgc{\textcolor{red}{P_{\rm g}}}
\newcommand{\ve}[1]{{\rm\bf {#1}}}
\newcommand{\Fconv}{F_{\rm conv}}
\def\ez{\ve{e}_z}
\def\ey{\ve{e}_y}
\def\ex{\ve{e}_x}
\def\ezero{\ve{e}_{0}}
\def\eplu{\ve{e}_{+1}}
\def\emin{\ve{e}_{-1}}
\def\definition{:=^{\!\!\!\!\!\!\!\textrm{def}}}
\newcommand{\bg}[1]{{\boldsymbol {#1}}}
\newcommand*\tavg[1]{%
  \hbox{%
    \vbox{%
      \hrule height 0.5pt % The actual bar
      \kern0.5ex%         % Distance between bar and symbol
      \hbox{%
        \kern-0.1em%      % Shortening on the left side
        \ensuremath{#1}%
        \kern-0.1em%      % Shortening on the right side
      }%
    }%
  }%
}
\title{Theoretical Astrophysics: Physics of Sun and Stars\\
Homework 3}
\author{P. K\"{a}pyl\"{a}, I. Mili\'{c}}
\date{\today}
%%%%
\begin{document}
\maketitle

\textbf{Deadline for this homework is \textbf{01}/07 23:59}

{\bf Problem 1:} For the stars in radiative equilibrium (all the energy is transported via radiation), the system of equations of the stellar structure can be investigated by dimensional analysis to get the so called homology relations. For example, we found the important relationship:

\begin{equation}
F \propto M^3
\end{equation}

Repeat the dimensional analysius and derive the results for fully convective stars, assuming:
\begin{align}
p &= K_a \rho^{\gamma_a} \nonumber \\
T &= K_a' P ^{(\gamma_a-1)/\gamma_a}
\end{align}

{\bf Problem 2:} 

\emph{Radiative equilibrium} is the assumption in which the transport of energy is done exclusively through radiation. Int this assumption, the total absorbed energy in a layer needs to be equal to the total emitted energy. Show that in stellar atmospheres this leads to: 

\begin{equation}
\int_0^{\infty} \kappa_\nu J_\nu d\nu = \int_0^{\infty} \kappa_\nu S_\nu d_\nu
\end{equation}

where $J$ is the mean intensity (averaged over angle), and $S$ is the source function (ratio of emission and absorption coefficients). What assumptions did we have to make in the process?


{\bf Problem 3:}

Kramer's opacity law, describes the free-free opacity of the completely ionized medium as: 

\begin{equation}
\kappa_ff \approx \frac{1}{2} \kappa_{ff,0} (1+X) \langle \frac{\rm Z^2}{A} \rangle \rho T^{-7/2}
\end{equation}
where $\kappa_{ff,0}$ is a constant equal to $7.5 \times 10^{21} {\rm m}^2 {\rm kg}^{-1}$ and 
\begin{equation}
\langle \frac{\mathcal Z^2}{A} \rangle  = \sum_i X_i \frac{\mathcal Z_i^2}{A_i} 
\end{equation}
where sum goes over all the relevant atoms, $X_i$ is the abundance, $Z$ is atomic number and $A$ mass number. 

Calculate the mean free path (inverse absorption coefficient, where absorption coeffient is the product of opacity and mass density), at the boundary between the core and the radiative zone ($\rho=10^4\,{\rm kg/m^3}$, $T=10^7$\,K) and at the photosphere (($\rho=10^{-7}\,{\rm kg/m^3}$, $T=6000$\,K)). 

What can you conclude about the assumption of local thermodynamic equilibrium (LTE) from this?

{\bf Useful physical constants}
\begin{itemize}
  \item $R_{\odot} = 696 \times 10^6\,{\rm m}$
  \item $M_{\odot} = 1.989 \times 10^{30}\,{\rm kg}$
  \item $L_{\odot} = 3.83 \times 10^{26}$~W
  \item $T^{\rm eff}_{\odot} = 5777\,{\rm K}$
  \item $1\,{\rm AU} = 1.496 \times 10^8\,{\rm km}$
  \item $c = 2.997 \times 10^8\,{\rm m/s}$
  \item $G = 6.674 \times 10^{-11}$~Nm$^2$/kg$^2$
  \item $k = 1.38\cdot10^{-23}$~J/K
  \item $m_{\rm H} = 1.67\cdot10^{-27}$~kg
  \item $h=6.626 \times 10^{-34}$~J~s.
  \item $k=1.38 \times 10^{-23}$~J/K.
\end{itemize}
\end{document}
